%
%
%
\chapter{Computation of the spin density wave propagator}
\label{ch: propagator}
%
%
%
In the present chapter the propagator of spin density waves should be computed up to the first order in pertubation theory.
The main goal of this masterthesis is the determination of the conductivity in the spin fermion model pertubated via umklapp scattering.
These processes are determined by spin density waves, see section \dots \todo{link to umklapp scattering}, why in the calculation of the conductivity the propagator of them is needed.

Firstly the free spin density wave propagtor is computed.
For that the equation of motion of Green functions is used. 
An good introduction of this method can be find in every textbook about quantum field theory in many body physics, but we want to recommended the book by Elk and Gasser \cite{Elk&Gasser}.
Afterwards the propagator is calculated up the first order using the pertubation theory in quantum field theory.
Finally the obtained propagator is transformed into the Matsubara frequency space.
An easy way to do this is using the Kramer-Kronig relations \eqref{eq:Kramer-Kronig relation}.
%
%
\section{The free propagator of spin density waves}
\label{sec: free propagator}
%
%
In \ref{sec: linear response theory} the linear response theory is established and the retarded susceptibility is introduced on this way.
A susceptibility describes the response of an operator because of an external pertubation.
This quantity is close related with the Green function or propagator of particles.
The only difference is that the operators and the expectation value of the Green function are represented in the Heisenberg picture comparing to the susceptibility, where them are represented with respect to the unpertubated system.

The equation of motion of Green functions is easy to get.
Therefore the Green function has only to be derivated with respect to the time.
The amazing result is that the equation of motion is equally for all typs (retarded, advanced and causal) of Green functions.
Only the boundary conditions are different.
The obtained equation of motion and the boundary conditions are transformed in Fourier space.
%
\begin{align}
	\omega \green{\mt{A}}{\mt{B}}_{\omega}^{j} = \expval*{\comm{\mt{A}}{\mt{B}}_{\eta}} + \green{\comm{\mt{A}}{\mt{H}}_{-}}{\mt{B}}_{\omega}^{j}
	\label{eq: algebraic equation chain}
\end{align}
%
where $j$ denotes the investigated Green function (retarded, advanced and causal) and $\omega$ represented that the Green function is in frequency space.
The double angle brackets symbolized the Green function of the operators A and B.
This equation is an algebraic equation or more precisely an infinite algebraic equation chain for the green function.
On the right hand side a new in general more complicated Green function appears.
For this one exists a new equation chain with a more complicted Green function on the right hand side and so on.
In the case of free propagators we are lucky.
The appearing Green function isn't really complicated, so that the initial Green function appears after one or two steps.
The same procedure can be done for the Green function in Matsubara time representation.
The result is similar to the one above, so that the frequency $\omega$ can be replaced with the Matsubara frequancy $i\omega_{n}$.
The simplicity and advantage of this method instead of other ones is that only the commutator relations of the (field) operators are needed.
Equation \eqref{eq: algebraic equation chain} is all we need to compute the free propagator of spin density waves.

The dynamic of free spin density waves is described by the Hamiltonian $\mt{H}_{\Phi}$, introduced in chapter \ref{ch: spin fermion model}.
Inserting $\mt{H}_{\Phi}$ in equation \eqref{eq: algebraic equation chain} yields the starting point of the following calculation.
Therefore the abbreviation $\green{\Phi_{\mu}}{\Phi_{\mu}}_{\omega}$ is introduced
The first operator is readed with the momentum argument $\vb{k}+\vb{G}$ 
In comparison the second operator is readed with the opposite one, $-\vb{k}-\vb{G}$.
The time argument is equal in both cases.
%
\begin{align}
	\omega \green{\Phi_{\mu}}{\Phi_{\mu}}_{\omega} &= 
		\expval{\comm{\Phi_{\mu}(\vb{k}+\vb{G},t)}{\Phi_{\mu}(-\vb{k}-\vb{G},t)}}
		+
		\green{\comm{\Phi_{\mu}(\vb{k}+\vb{G},t)}{\mt{H}_{\Phi}}}{\Phi_{\mu}}_{\omega}
		\label{eq: equation chain SDW}
\end{align}
%

The bosonic commutator relations are given in equation (\dots\todo{link zu bosonischen Vertauschungsrelationen}).
Like it is shown the only non-vanishing commutator relation is this one between the bosonic field operator and them canonical momentum operator.
Therefore on the right hand side of \eqref{eq: first item of equation chain} the inhomogeneity is vanished.
For computing the Green function on the same side the Hamiltonian $\mt{H}_{\Phi}$ in equation \todo{link zu $H_{\Phi}$} is used.
The commutator is given by
%
\begin{align}
	\comm{\Phi_{\mu}(\vb{k}+\vb{G},t)}{\mt{H}_{\Phi}} &= 
		-\frac{1}{2\epsilon} 
		\sum\limits_{\vb{P}} 
		\int_{\vb{p}}
		\comm{\Phi_{\mu}(\vb{k}+\vb{G},t)}{\pi_{\lambda}(\vb{p}+\vb{P},t_{1}) \pi_{\lambda}(-\vb{p}-\vb{P},t_{1})}
	\notag \\
	\Leftrightarrow\ \comm{\Phi_{\mu}(\vb{k}+\vb{G},t)}{\mt{H}_{\Phi}} &= 
		-\frac{1}{2\epsilon} 
		\sum\limits_{\vb{P}} 
		\int_{\vb{p}} \bigg[
			\pi_{\lambda}(\vb{p}+\vb{P},t_{1}) \comm{\Phi_{\mu}(\vb{k}+\vb{G},t)}{\pi_{\lambda}(-\vb{p}-\vb{P},t_{1})}
			\notag \\& \hspace{2cm}
			+
			\comm{\Phi_{\mu}(\vb{k}+\vb{G},t)}{\pi_{\lambda}(\vb{p}+\vb{P},t_{1})} \pi_{\lambda}(-\vb{p}-\vb{P},t_{1})
		\bigg]
	\notag \\
	\Leftrightarrow\ \comm{\Phi_{\mu}(\vb{k}+\vb{G},t)}{\mt{H}_{\Phi}} &= 
		-\frac{i}{\epsilon} \pi_{\mu}(\vb{k}+\vb{G},t)
\end{align}
%
where in the beginning the sum over $\lambda$ is implied.
Inserting the obtained result of the commutator in equation \eqref{eq: equation chain SDW} yields the connection between the initial and the new Green function.
%
\begin{align}
	\omega \green{\Phi_{\mu}}{\Phi_{\mu}}_{\omega} &= 
		-\frac{i}{\epsilon} \green{\pi_{\mu}}{\Phi_{\mu}}_{\omega}
	\label{eq: first item of the chain}
\end{align}
%
Equally to the initial Green function an algebraic equation chain can be established for the new Green function.
%
\begin{align}
	\omega \green{\pi_{\mu}}{\Phi_{\mu}}_{\omega} &= 
		\expval{\comm{\pi_{\mu}(\vb{k}+\vb{G},t)}{\Phi_{\lambda}(-\vb{k}-\vb{G},t)}}
		+
		\green{\comm{\pi_{\mu}(\vb{k}+\vb{G},t)}{\mt{H}_{\Phi}}}{\Phi_{\mu}}_{\omega}
\end{align}
%
Now the same things like above are to do.
The inhomogeneity is given by the commutator relations \todo{link to commutator relations}.
Incomparison to the case above this time the commutator dosen't vanish but yields $-i$.
For the Green function on the right hand side again the commutator has to be calculated, which yields $\comm{\pi_{\mu}(\vb{k}+\vb{G},t)}{\mt{H}_{\Phi}} = i \big((\vb{k}+\vb{G})^{2} + r\big) \Phi_{\mu}(\vb{k}+\vb{G},t)$.
In total we obtain the relation
%
\begin{align}
	\omega \green{\pi_{\mu}}{\Phi_{\mu}}_{\omega} = 
		-i + i\Big((\vb{k}+\vb{G})^{2} + r \Big) \green{\Phi_{\mu}}{\Phi_{\mu}}_{\omega}.
		\label{eq: second item of the chain}
\end{align}
%
Again on the right hand side a new Green function appears.
This time the new Green function is well known, because it's the initial one.
Both equations, \eqref{eq: first item of the chain} and \eqref{eq: second item of the chain}, are an equation system, where the Green function of $\pi$ and $\Phi$ can be eliminated.
The easiest way doing this is to multiply equation \eqref{eq: first item of the chain} with $\omega$ and inserting \eqref{eq: second item of the chain} in the obtained relation.
%
\begin{align}
	\green{\Phi_{\mu}}{\Phi_{\lambda}}_{\omega} = \sum\limits_{\vb{G}} \frac{1}{(\vb{k}+\vb{G})^{2} + r - \xi^{-2}} =: \mathcal{D}_{\mu}^{(0)}(\vb{k},\omega)
	\label{eq: free spin density wave propagator}
\end{align}
%
where the invers squared correlation length $\xi^{-2} = \epsilon \omega^{2}$ is introduced.
The free propagator exhibits a periodicity respective to the Brillouin zone.
This condition is used in the calculation of the conductivity below. \todo{say a little bit more about that}
%
%
\section{The damped spin density wave propagator}
\label{sec: damped propagator}
%
%
In the previous section the free propagator of spin density waves is computed.
Beside the free dynamics the spin fermion model considers an interaction between electrons living on different Fermi surfaces, where the interaction is originated by the spin density waves.
Therefore the propagation of the spin density waves is damped.
This damping should be considered in the propagator, which is done by doing pertubation theory.

Because the damping is originated by the interaction between electrons the free electron propagator is needed, which can be calculated in the same way as the free spin density propagator.
This handwork shouldn't be done here explicitly.
The free electron propagator is given by
%
\begin{align}
	 \green{\Psi_{\alpha}}{\Psi_{\alpha}^{\dag}}_{\omega} = \sum\limits_{\vb{G}} \frac{1}{\omega - \epsilon_{\alpha}(\vb{k}+\vb{G})} =: \mathcal{G}_{\alpha}^{(0)}(\vb{k},\omega), 
	 \label{eq: free electron propagator}
\end{align}
%
where $\alpha = \mt{a,b}$ which denotes the Fermi surface of the respective electrons.
The damped spin density wave propagator is computed using the usually method of pertubation theory in quantum field theory.
The full spin density wave propagator is given by
%
\begin{align}
	\mathcal{D}_{\mu}(\vb{k}, t-t') = -i \expval{\mathcal{T}_{t} \mt{U}(\infty, -\infty) \Phi_{\mu}(\vb{k}+\vb{G},t) \Phi_{\mu}(-\vb{k}-\vb{G},t')}_{0}^{\mt{con}}
	\label{eq: full spin density wave propagator}
\end{align}
%
where $\mathcal{T}_{t}$ is the time ordering operator, which orders all contained operators in the right time order.
The index $0$ denotes that the expectation value is performed with respect to the unpertubated Hamiltonian.
The interaction is only incorporated through the time evolution operator $\mt{U}$ which is given by
%
\begin{align}
	\mt{U}(t,t') = \exp\bigg(-i\int_{t'}^{t} \dd{t_{1}} \mt{H}_{\mt{int}}(t_{1})\bigg).
	\label{eq: time evolution operator}
\end{align}
%
The second index "con" at the expectation value denotes that only connected diagrams are considered.
In the so called link cluster theorem it is proven that all disconnected diagrams are canceled with the vacuum diagrams, see \cite{Nolting} for it.
All these connected diagrams can be simplified a little bit more.
There exist diagrams which are contained only diagrams of a lower order in a specific way.
It is possible to build these diagrams by multipling diagrams of lower orders.
Therefore a new object is introduced, called self energy $\Pi$, which contains all irreducible connected diagrams.
The self energy offers the oppertunity to write the full Green function as a Dyson equation.
%
\begin{align}
	\mathcal{D}_{\mu} = \mathcal{D}_{\mu}^{(0)} + \mathcal{D}_{\mu}^{(0)} \Pi_{\mu} \mathcal{D}_{\mu}
	\qquad \Rightarrow\ \qquad
	\mathcal{D}_{\mu} = \frac{1}{\big(\mathcal{D}_{\mu}^{(0)}\big)^{-1} - \Pi_{\mu}}
	\label{eq: Dyson equation}
\end{align}
%
The self energy is a complex quantity in general.
Splitting them in a real and imaginary part the real part is a correction to the energy and the imaginary part is interpreted as a life time.
A life time means nothing else that the partile is damped.
The goal of the following calculation is to compute the imaginary part of the self energy.
To get a feeling how the self energy looks like or in other words which diagrams has to be considered the full propagator in \eqref{eq: full spin density wave propagator} is investigated.

The time evolution operator is expanded up to the second order.
The zeroth order yields the free propagator which is calculated in the previous section.
Further the first order vanishes.
The interaction Hamiltonian $H_{\Psi\Phi}$ contains one bosonic field operator and therefore combining with the two other bosonic operators this yields an expectation value of three bosonic operators.
Wick's theorem says that the expectation value of an odd number of operators is always zero.
The reason is that it's impossible to get an term where only contractions are contained.
Having an odd number of operators a normal product exist in every term.
Taking the equillibrium expectation value of a normal product, it's zero by definition.

The first not vanishing contribution appears at the second order, because the interaction Hamiltonian $H_{\Psi\Phi}$ contributes twice, thus the number of operators is even in both cases, fermionic and bosonic.
In the fermonic case four expectation values appear, where those ones have a little bit different structure like the usually known expectation values of fermionic operators.
The feature of them is that two fermionic operators are connected by a Pauli matrix, which prohibit the use of Wick's theorem or any rearrange of the operators.
The expectation values looks like
%
\begin{align}
	\expval{\mathcal{T}_{t}\ \Psi_{\alpha}^{\dag}(\vb{p}_{4},t_{2}) \cdot \sigma_{\lambda'} \cdot \Psi_{\beta}(\vb{p}_{3},t_{2}) \cdot \Psi_{\gamma}^{\dag}(\vb{p}_{2},t_{1}) \cdot \sigma_{\lambda} \cdot \Psi_{\beta}(\vb{p}_{1},t_{1})}_{0},
	\label{eq: structure of fermionic expval}
\end{align}
%
where $\alpha,\beta,\gamma,\delta \in \{\mt{a},\mt{b}\}$ with the property that always two greek letters have to be an "a" and the other two ones a "b".
Fierz identity offers an oppertunity to eliminate the Pauli matricies.
With the aid of Fierz identity a product of the components of two Pauli matricies can be rewriten as a relation of Kronecker symbols.
%
\begin{align}
	\sum\limits_{\mu = 1}^{3} \sigma_{ij}^{\mu} \sigma_{kl}^{\mu} = 2 \delta_{il} \delta_{jk} - \delta_{ij} \delta_{kl}
	\label{eq: Fierz identity}
\end{align}
%
For acting the Fierz identity the product of field operators and Pauli matrix has to be writen in component representation.
Then the identity can be use without any doubt and the Kronecker symbols allows us to rewrite the operators without component representation.
Doing this we have to rivet on the first term in \eqref{eq: Fierz identity}, because the order of the operators is rearranged.
Therefore the operators has to be commuted  with yields a $\delta$-distribution with respect to the momentum, see \eqref{appeq: general expectation value after Fierz identity}.
The exact calculation is done in the appendix \ref{app: Fierz identity}.

Each obtained expectation values contains two operators of a-electrons\footnote{a-electrons denotes electrons living on the Fermi surface labeled with a. In comparison b-electrons are electrons on the Fermi surface labeled with b. Both Fermi surfaces are rotated by $\flatfrac{\pi}{2}$ and shifted by $(\pm\pi,\pm\pi)$.} and two operators of b-electrons, so that the expectation value can be seperated.
One half of them is so constructed that both annihiliation operators are acting on a-electrons, for example.
These kinds of expectation values are surely zero
The other half is "normaly" constructed so that one annihiliation and creation operator is acting on a-electrons.
The same is surely valid for b-electrons.

Bringing the remained operators in the order that all annihilaition operators stands on the left side of the creation operators yields a $\delta$-distribution for each commutation, so that in total each term containts two $\delta$-distributions.
The obtained expectation value is shown in equation \eqref{appeq: expectation value of second order correction} in the appendix.

Inserting equation \eqref{appeq: expectation value of second order correction} for each of the four expectation values two of the four momentum integrals and sums can be performed.
The remaining expression for $\mathcal{D}$ in second order pertubation theory have the form
%
\begin{align}
	\mathcal{D}_{\mu}^{(2)}(\vb{k}, \omega) &= 
		(-i)^{3} \lambda^{2}
		\int\limits_{-\infty}^{\infty} \dd{t_{1}} \dd{t_{2}}
		\sum\limits_{\vb{P}_{1} \vb{P}_{2}} \int_{\vb{p}_{1}} \int_{\vb{p}_{2}}
		\notag \\ &\times		
		\expval{
			\mathcal{T}_{t} 
			\Phi_{\lambda'} (\tilde{\vb{p}}_{2}-\tilde{\vb{p}}_{1},t_{2}) 
			\Phi_{\lambda} (\tilde{\vb{p}}_{1}-\tilde{\vb{p}}_{2},t_{1}) 
			\Phi_{\mu}(\tilde{\vb{k}},t) 
			\Phi_{\mu}(-\tilde{\vb{k}},t')
		}_{0}
		\notag \\
		&\times
		\bigg(
		\expval{\mathcal{T}_{t} \Psi_{\mt{a}}(\tilde{\vb{p}}_{2},t_{1}) \Psi_{\mt{a}}^{\dag}(\tilde{\vb{p}}_{2},t_{2})}_{0}
		\expval{\mathcal{T}_{t} \Psi_{\mt{b}}(\tilde{\vb{p}}_{1},t_{2})	\Psi_{\mt{b}}^{\dag}(\tilde{\vb{p}}_{1},t_{1})}_{0}
		\notag \\
		&+
		\expval{\mathcal{T}_{t}	\Psi_{\mt{b}}(\tilde{\vb{p}}_{2},t_{1}) \Psi_{\mt{b}}^{\dag}(\tilde{\vb{p}}_{2},t_{2})}_{0}
		\expval{\mathcal{T}_{t}	\Psi_{\mt{a}}(\tilde{\vb{p}}_{1},t_{2})	\Psi_{\mt{a}}^{\dag}(\tilde{\vb{p}}_{1},t_{1})}_{0}
		\bigg)
\end{align}
%
where the abbreviation $\tilde{\vb{k}} = \vb{k}+\vb{G}$ and $\tilde{\vb{p}}_{i} = \vb{p}_{i} + \vb{P}_{i}$ with $i=1,2$ is introduced.
In the case of the bosonic expectation value the usually used Wick theorem is utilized.
Wick's theorem yields three possibile contractions in the corresponding case, where one of them isn't contributed, because it's yielded disconnected diagrams.
The remaining two contractions generate four diagrams in total, which are depicted in figure \dots \todo{picture of the diagrams}.
These diagrams differentiate only in two points.

On the one hand the acting point of the bosonic lines is changed comparing the first two and last two diagrams.
On the other hand the direction of the fermionic lines is interchanged between the first two diagrams.
In the first diagram an a-electron is annihiliated and a b-electron is created at $t_{1}$.
Comparing the second diagram, where an a-electron is created and a b-electron is annihilated at $t_{1}$.
This is valid for the last two diagrams too.
All four diagrams are closly connected with each other, because all other diagrams can be generated out of one diagram by interchanging the acting point of the bosonic lines or the direction of the fermionic lines.
Therefore all four diagrams yield the same contribution, where it is enough to compute one of them an multiply by four.
%
\begin{align}
	\mathcal{D}_{\mu}^{(2)}(\vb{k}, t-t') &= 
		(-i)^{3} 4 \lambda^{2}
		\int\limits_{-\infty}^{\infty} \dd{t_{1}} \dd{t_{2}}
		\sum\limits_{\vb{P}} \int_{\vb{p}}
		\notag \\ &\times
		\expval{\mathcal{T}_{t} \Phi_{\mu} (\tilde{\vb{k}},t_{2}) \Phi_{\mu}(-\tilde{\vb{k}},t')}_{0}	
		\expval{\mathcal{T}_{t} \Phi_{\mu} (-\tilde{\vb{k}},t_{1}) \Phi_{\mu}(\tilde{\vb{k}},t)}_{0}
		\notag \\ &\times
		\expval{\mathcal{T}_{t} \Psi_{\mt{a}}(\tilde{\vb{p}}-\tilde{\vb{k}},t_{1}) \Psi_{\mt{a}}^{\dag}(\tilde{\vb{p}}-\tilde{\vb{k}},t_{2})}_{0}
		\expval{\mathcal{T}_{t} \Psi_{\mt{b}}(\tilde{\vb{p}},t_{2}) \Psi_{\mt{b}}^{\dag}(\tilde{\vb{p}},t_{1})}_{0}
		\label{eq: spin density wave propagator second order correction}
\end{align}
%

where the momentum $\vb{p}_{1}$ and $\vb{P}_{1}$ is relabeled with $\vb{p}$ and $\vb{P}$, respectivily.
Accordingly we write $\tilde{\vb{p}}$ instead of $\tilde{\vb{p}}_{1}$.
In comparison to the Dyson equation \eqref{eq: Dyson equation} the fermionic bubble is identified with the self energy $\Pi_{\mu}$, where the bubble diagram surely only represented the zeroth order of the self energy.
The self energy in zeroth order is given by
%
\begin{align}
	\Pi_{\mu}^{(0)}(\vb{k}, \omega) = 
		i 
		\sum\limits_{\vb{P}}
		\int\limits_{|p| \leq |p_{\mt{F}}} \frac{\dd[2]{\vb{p}}}{(2\pi)^{2}} 
		\int\limits_{-\infty}^{\infty} \frac{\dd{\epsilon}}{2\pi}\,
		\mathcal{G}_{\mt{a}}^{(0)}(\vb{p}+\frac{\vb{k}}{2}, \epsilon+\frac{\omega}{2})
		\mathcal{G}_{\mt{b}}^{(0)}(\vb{p}-\frac{\vb{k}}{2}, \epsilon-\frac{\omega}{2}).
\end{align}
%
In comparison to equation \eqref{eq: spin density wave propagator second order correction} the self energy is represented in frequency space.
Further the outer momentum and frequency is shifted by the half of itself, so that both arguments of the fermionic propagators are symmetricly.
How we see in equation \eqref{eq: free electron propagator} the free electron propagator contains the dispersion relation $\epsilon_{\alpha}(\vb{p}+\vb{P})$ of the respective electrons.
In our considered spin fermion model only electrons near the Fermi surface interacte with each other interfered by spin density waves, which means that the momentum transfer is small.
Under this condition the dispersion relation can be expanded near the Fermi surface.
%
\begin{align}
	\epsilon_{\mt{a}}(\vb{p} + \vb{P} + \frac{\vb{k} + \vb{G}}{2}) &= 
		\frac{\big(p_{x} + P_{x} + \frac{k_{x} + G_{x}}{2}\big)^{2}}{2m_{1}} 
		+ 
		\frac{\big(p_{y} + P_{y} + \frac{k_{y} + G_{y}}{2}\big)^{2}}{2m_{2}} 
		\notag \\ 
	\Leftrightarrow\ \epsilon_{\mt{a}}(\vb{p} + \vb{P} + \frac{\vb{k} + \vb{G}}{2}) &=
		\frac{(p_{x} + P_{x})^{2}}{2m_{1}} + \frac{1}{2} \frac{p_{x}+P_{x}}{m_{1}}(k_{x}+G_{x}) + \frac{(k_{x}+G_{x})^{2}}{8m_{1}}
		\notag \\ &+
		\frac{(p_{y} + P_{y})^{2}}{2m_{2}} + \frac{1}{2} \frac{p_{y}+P_{y}}{m_{2}}(k_{y}+G_{y}) + \frac{(k_{y}+G_{y})^{2}}{8m_{2}}
		\notag \\
	\Leftrightarrow\ \epsilon_{\mt{a}}(\vb{p} + \vb{P} + \frac{\vb{k} + \vb{G}}{2}) &\approx
		\xi_{\mt{a}} + \frac{1}{2} \vb{v}_{\mt{a,F}}(\vb{k} + \vb{G}) + \mu
\end{align} 
%
where the quadratic term with respect to $\vb{k}$ is negatable, because the bosonic transfered momentum is small.
Further the velocity $\vb{v}_{a}$ of the a-electrons is indrofuced, where the elelctron velocity can be approximated with the Fermi velocity of the corresponing Fermi surface, because only electrons near the Fermi surface are considered.
Besides the dispersion relation \dots \todo{link zur dispersion im spin fermion model} of the a-electrons is used.
The same procedure is done for the b-electrons.
Finally the normalized momentum vector $\vb{n} = \frac{\vb{p}+\vb{P}}{|\vb{p}+\vb{P}|}$ is introduced.
The Fermi velocity of the a-electrons is then given by $\vb{v}_{\mt{a,F}} = v_{\mt{a,F}} \vb{n}$ for example.
The scalar product between the normalized momentum vector $\vb{n}$ and the bosonic spin density wave vector $\vb{k} + \vb{G}$ is rewriten as the magnitude of them multiplied with $\cos(\vartheta)$, where $\vartheta$ is the angle between both.

In the investigated spin fermion model the electrons on different Fermi surfaces only interacte on so called hot spots like we have it introduced in chapter \ref{ch: spin fermion model}.
The energy on the hot spots is equal and the the magnitudes of the Fermi velocities are equal too.
%
\begin{align}
	\xi := \xi_{\mt{a}} = \xi_{\mt{b}} \qquad v_{\mt{F}} := v_{\mt{a,F}} = v_{\mt{b,F}}
\end{align}
%
Consider that the direction of the velocities haven't been equal, otherwise the angle $0$ or $\pi$ and the imaginary part of $\Pi_{\mu}$ is zero.
Using these assumptions the self energy is given by
%
\begin{align}
	\Pi_{\mu}^{(0)}(\vb{k}, \omega) &= 
		i \nu_{\mt{F}}
		\sum\limits_{\vb{G}}
		\int\limits_{0}^{\pi} \dd{\vartheta}
		\int\limits_{\xi \leq \xi_{\mt{F}}} \dd{\xi}
		\int\limits_{-\infty}^{\infty} \frac{\dd{\epsilon}}{2\pi}
		\notag \\ &\times
		\frac{1}{\epsilon + \frac{\omega}{2} - \xi - \frac{1}{2} v_{\mt{F}} |\vb{k} + \vb{G}| \cos(\vartheta) + i \eta \sign(\epsilon + \frac{\omega}{2})}
		\notag \\ &\times
		\frac{1}{\epsilon - \frac{\omega}{2} - \xi + \frac{1}{2} v_{\mt{F}} |\vb{k} + \vb{G}| \cos(\vartheta) + i \eta \sign(\epsilon - \frac{\omega}{2})},
\end{align}
%
where the two dimensional momentum integral is firstly transformed in plane polar coordinates and than in an energy integral over the density of states.
Because only electrons near the Fermi surface are considered the density of states can be approximated with the constant one $\nu_{\mt{F}} := \nu(\xi_{\mt{F}})$ at the Fermi surface.
The energy integral is certainly limited by the Fermi energy $\xi_{\mt{F}}$.

The further investigation is been starting with the computation of the frequency integral.
Therefore the integral over $\epsilon$ is transformed into a complex contour integral.
The contour $\Gamma$ is chosen in two different ways.
In both case the contour starts along the real axis.
According to the singularities of the integrand the countour is closed in the upper or lower half plane via a semicircle with radius infinity.
In both cases the contribution of the semicircle is zero because the integrand goes like $\flatfrac{1}{\epsilon}$.
The non-contributing of the semicircle ensures the equality between the integral along the real axis and the complex contour integral.
The investigated integrand occupies two singularities at
%
\begin{align}
	\epsilon_{1} := \xi - \frac{1}{2}\big(\omega - v_{\mt{F}} |\vb{k} + \vb{G}| \cos(\vartheta)\big) 
	\qq{and}
	\epsilon_{2} := \xi + \frac{1}{2}\big(\omega - v_{\mt{F}} |\vb{k} + \vb{G}| \cos(\vartheta)\big),
\end{align}
%
where both singularities are of first order which allows us to use Cauchy's integral formula.
According to the signum function in both denominators the poles are located in the upper or lower plane.
So in total there are four different possible constitutions.
On the one hand both singularities can be located in the lower or upper complex plane
On the other hand one pole can be in the upper plane and the other one can be in the lower plane, and vice versa.






































