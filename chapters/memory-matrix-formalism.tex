%
%
\chapter{Memory-Matrix-Formalsim}
\label{ch: memory-matrix-formalism}
%
%
%
\section{Motivation}
\label{sec: motivation}
%
%
%
A physicist is always interested in the beaviour and time evolution of the observables of the investigates system.
In the middle of the last century many physicists worked on the understanding and mathematical description of one physical process, the Brownian motion.
On stochasical theory of these certain physical process is based on the Langevin equation
%
\begin{align}
	\pdv{t} \mt{A}(t) -\mt{F}_{\mt{ex}}(x,t) + \gamma \cdot \mt{A}(t) = f(t),
	\label{eq: Langevin equation}
\end{align}
%
where $\mt{A}(t)$ is some dynamical observable and $f(t)$ is a random force like white noise for example.
The origin of the second term on the left hand side is some external force result from a coupling between $\mt{A}(t)$ and some external potential.
The third term on the left hand side is a damping or friction term.
Now let us assume it's possible to seperate equation \eqref{eq: Langevin equation} into two parts.
The first part, called $f_{1}$, is a functional of the dynamical observable $\mt{A}(t')$, where $t_{0} \leq t' \leq t$, so that this part is depending on the history of A.
The second part $f_{2}$ should be depending on all other degrees of freedom.
Now $f_{1}$ is expanded up to the linear order and all terms of higher order and the part $f_{2}$ are summerized to the quantity $F(t)$.
The result is a linearized form of the Langevin equation
%
\begin{align}
	\pdv{t} \mt{A}(t) = \int\limits_{t_{0}}^{t} \dd{t'} \mathcal{C}(t-t') \mt{A}(t') + F(t),
	\label{eq: linearized Langevin equation}
\end{align}
%
where $\mathcal{C}$ is a correlation function and $\mt{A}(t')$ is the deviation of the invariant part of the Hamiltonian.
For large time scales the deviation should be vanish, so the time-integral over $\mt{A}(t')$ should be become zero.
For simplification the origin of the time axis is moved to $t_{0}$.
In general the Laplace transformation of a function is given by
%
\begin{align}
	\mathcal{L}\big\{\mt{A}(t)\big\} = \mt{A}(s) = \int\limits_{0}^{\infty} \dd{t} \mt{A}(t) e^{-st}.
	\label{eq: Laplace transformation real axis}
\end{align}
%
Using the Laplace transformation equation \eqref{eq: linearized Langevin equation} becomes a algebratic equation of motion.
The solution of this equation is 
%
\begin{align}
	\mt{A}(t) = \Xi(t) \cdot \mt{A}(0) + \mt{A}'(t) \hspace{1cm} \mt{with} \hspace{1cm} \mt{A}'(t) = \int\limits_{0}^{t} \dd{t'} \Xi(t-t') F(t'),
	\label{eq: splitted observable}
\end{align}
%
where the function $\Xi(t)$ is defined by the Laplace transformation of $\Xi(s) = [s-\mathcal{C}(s)]^{-1}$ and $\mathcal{C}(s)$ is the Laplace transformtion of the correlation function $\mathcal{C}(t)$.
The main result of equation \eqref{eq: splitted observable} and the motivation for the following introduced memory-matrix-formalism is the splitting of the dynamical observable $\mt{A}(t)$ into two parts.

For the first term on the right hand side the only time-dependence is adverted through the correlation function $\mathcal{C}$, which is clear consider the definition of $\Xi$.
This term included the linear contributions of $\mt{A}(t)$ by construction.
These ones are the mostly important contributions to the time evolutaion of an observable, because they are secular.
In contrast the second term on the right hand side is the convolution between the function $\Xi(t-t')$ and the function $\mt{F}(t')$.
The latter summerize all the non-linear effects, fluctuations and intital transient processes, which are all effects with a small lifetimes in contrast with the secular effects.
Therefore these effects shouldn't have large influences on the time evolution of an observable, always large time scales in mind.

Beside the physical interpretation a simple geometrical and mathematical one is very usefull.
Let us assume a vector space ana the observable should be a vector in this vector space.
Then the secular term is a projection on the A-axis and the non-secular term is aquivalent to a vector perpendicular to the A-axis.
The memory-matrix-formalism take up this simple interpretation of equation \eqref{eq: splitted observable} and put it in a general and exact form, so that it can be used classicaly and quantum mechanicaly.
%
%
%
\section{Linear Response Theory}
\label{sec: linear response theory}
%
%
%
Before the derivation of the memory-matrix-formalism can be started some ground work is to do.
This section begins with a short reminder of the kubo formula. %also known as linear response theory.
After that the Kubo relaxation function are introduced and some important relations between there and the retarded susceptibility $\chi$ are derivated.
In the last section finally the splitting of $\chi$ in a real and an imaginary part are dicussed.
%
%
\subsection{Kubo formula}
\label{subsec: kubo formula}
%
%
Consider a system in equilibrium represented by the Hamiltonian $H_{0}$.
At an arbitrary time $t'$ a pertubation is switched on, where the pertubation is given by the Hamiltonian $H_{1} = - B \cdot F(t)$, so that $H(t) = H_{0} + H_{1}$ is the full Hamiltonian.
Thereby $B$ is an operator by which the pertubation is coupled on the system and $F(t)$ is a function determining the time evolution of the pertubation.
It is assumed that $F(t) = 0$ for $t<t'$ so that the system is in thermal equilibrium for all these times.

The physical interest is existed in the question how does an observable $\mt{A}$ react on the pertubation switched on at $t'$.
The answer is given by the thermodynamical expectation value of the operator corresponding to the observable $\mt{A}$
%
\begin{align}
	\expval{\mt{A}}(t) := \Tr{\rho_{\mt{S}}(t) \mt{A}_{\mt{S}}} = \Tr{\rho_{\mt{I}}(t) \mt{A}_{\mt{I}}},
	\label{eq: thermodynamical expectation value}
\end{align}
%
where the label S and I stand for the Schrödinger and Interaction picture, respectivily.
The equality of the expectation value in the different representations is shown by the invariance of the trace with resprect to cycle permutation.
The transformation into the interaction picture is very usefull what we will see after the next step below.
In quantum mechnics the time evolution of the density operator is determined by the von Neumann-equation.
%
\begin{align}
	\dv{t} \rho_{\mt{S}}(t) = -\frac{i}{\hbar} \comm{H(t)}{\rho_{\mt{S}}(t)} \hspace{0.5cm}\Leftrightarrow\hspace{0.5cm} \dv{t} \rho_{\mt{I}}(t) = -\frac{i}{\hbar} \comm{H_{1}}{\rho_{\mt{I}}(t)}
	\label{eq: von Neumann-equation}
\end{align}
%
The equation is also transformed into the interaction picture, which doesn't change the structure itself but the density operator deponds only on the Hamiltonian $H_{1}$ now.
Integrating and using the boundary condition that the system is in thermal equilibrium at $t \to -\infty$ equation \eqref{eq: von Neumann-equation} is resulted in a integrable equation for the density operator.
%
\begin{align}
	\rho_{\mt{I}}(t) = \rho_{0} + \frac{i}{\hbar} \int\limits_{-\infty}^{t} \dd{t'} \comm{B_{\mt{I}}(t')}{\rho_{\mt{I}}(t')}F(t')
	\label{eq: integrable form of von Neumann-equation}
\end{align}
%
Jet it is clear why the interection picture is used.
The integrand depends on the Hamiltonian of the pertubation only in linear order which is a perfect starting point for a iterativ solution procedure.
Starting with the zeroth order the density operator is trivially the density operator at thermical equilibrium.
Inserting the zeroth order on the right hand side of equation \eqref{eq: integrable form of von Neumann-equation} yield the first order of the density operator, a.\,s.\o.
In linear response theory the iteration is cutted off after the first order.
Inserting this in equation \eqref{eq: thermodynamical expectation value} and defining the dynamical susceptibility 
%
\begin{align}
	\chi_{\mt{AB}}(t-t') = \frac{i}{\hbar} \Theta(t-t') \expval{\comm{A_{\mt{I}}(t-t')}{B_{\mt{I}}(0)}}_{H_{0}}
	\label{eq: dynamical susceptibilty}
\end{align}
%
yield the Kubo formula
%
\begin{align}
	\delta\expval{\mt{A}(t)} := \expval{\mt{A}}(t) - \expval{\mt{A}(t)}_{H_{0}} \approx \int\limits_{-\infty}^{\infty} \dd{t'} \chi_{\mt{AB}}(t-t') F(t'),
	\label{eq: Kubo formula}
\end{align}
%
where the label $H_{0}$ means that the expactation value is taken with respect to the unpertubated Hamiltonian.
We see that the deviation of the observable A caused by the pertubation is given by the convolution of the dynamical suszeptibilty $\chi_{\mt{AB}}(t-t')$ and the time evolution function $F(t)$.\todo{noch sch\"oner schreiben}
%
%
\subsection{Kubo relaxation function}
\label{subsec: Kubo relaxation function}
%
%
After a general equation for the deviation of an observable A from the equilibrium value was established, we want to investigate a certain kind of pertubation.
Let us assume $F(t) = \Theta(-t) \cdot F \cdot e^{-s\tau}$ the time evolution function of a pertubation, which is switched on adiabatically at $t=-\infty$ and switched off at $t=0$.
Inserting this in equation \eqref{eq: Kubo formula} and substituting $\tau = t-t'$ yield $\delta\expval{\mt{A}(t)} = \Phi_{AB}(t) \cdot F e^{st}$ with the Kubo relaxation function
%
\begin{align}
	\Phi_{\mt{AB}}(t) = \frac{i}{\hbar} \lim\limits_{s \to 0} \int\limits_{t}^{\infty} \dd{\tau} \expval{\comm{\mt{A}_{\mt{I}}(\tau)}{\mt{B}_{\mt{I}}(0)}}_{0} e^{-s\tau}.
	\label{eq: Kubo relaxation function}
\end{align}
%
The arising $\Theta$-distributions determine the lower limit of the intergal to $t$.
For a more detailed derivation of the Kubo relaxation function see \cite{Schwabl} or \cite{Schwabl2}.
It's not really surprisingly that the Kubo relaxation function and the dynamical susceptibility are closely connected, because the first is derivated out of the latter one.
However there exist three very important relations between them both, which are 
%
\begin{enumerate}
	\item $\begin{aligned} \chi_{\mt{AB}}(t) = -\Theta(t) \dv{t} \Phi_{\mt{AB}}(t) \label{eq: relation 1 between Phi and chi} \end{aligned}$
	\item $\begin{aligned} \Phi_{\mt{AB}}(t = 0) = \chi_{\mt{AB}}(\omega = 0) \label{eq: relation 2 between Phi and chi} \end{aligned}$
	\item $\begin{aligned} \Phi_{\mt{AB}}(\omega) = \frac{1}{i\omega}\big[\chi_{\mt{AB}}(\omega) - \chi_{\mt{AB}}(\omega = 0)\big]. \label{eq: relation 3 between Phi and chi} \end{aligned}$
\end{enumerate}
%
The evidence of these tree relations are shown in the appendix \ref{app: properties of the Kubo relaxation function}.
For the later deviation of the memory-matrix-formalism it's more usefull to write the Kubo relaxation function in another, not so intuitivly form.
The goal of the rewriting is to get the expectation value in a form with no commutator and to do this two identities are needed.
The first one is
%
\begin{align}
	\expval{\comm{\mt{A}(t)}{\mt{B}(t')}} &= \frac{1}{Z} \Tr{\comm{\rho}{\mt{A}(t)} \mt{B}(t')},
	\label{eq: identity expectation value}
\end{align}
%
where the invariance of the expactation value with respect to cycling permutation is used.
The second one is the Kubo-identity.
Thereby the main idea is to used the analogy of the exponential functions to the time evolution of an operator.
%
\begin{align}
	i \comm{\rho}{\mt{A}(t)} &= i \Big[\rho \mt{A}(t) - \mt{A}(t) \rho\Big]
	\notag \\
	\Leftrightarrow\ i \comm{\rho}{\mt{A}(t)} &= i \Big[\rho \mt{A}(t) - e^{-\beta H} e^{\beta \mt{H}} \mt{A}(t) e^{-\beta \mt{H}}\Big]
	\notag \\
	\Leftrightarrow\ i \comm{\rho}{\mt{A}(t)} &= -i \rho \int\limits_{0}^{\beta} \dd{\lambda} \dv{\lambda} e^{\lambda \mt{H}} \mt{A}(t) e^{-\lambda \mt{H}}
	\notag \\
	\Leftrightarrow\ i \comm{\rho}{\mt{A}(t)} &= -i \rho \int\limits_{0}^{\beta} \dd{\lambda} \bigg[\mt{H} e^{i\tilde{\lambda} \mt{H}/\hbar} \mt{A}(t) e^{-i\tilde{\lambda} \mt{H}/\hbar} - e^{i\tilde{\lambda} \mt{H}/\hbar} \mt{A}(t) e^{-i\tilde{\lambda} \mt{H}/\hbar} \mt{H}\bigg]
	\notag \\
	\Leftrightarrow\ i \comm{\rho}{\mt{A}(t)} &= -i \rho \int\limits_{0}^{\beta} \dd{\lambda} \comm{\mt{H}}{\mt{A}(t+\tilde{\lambda})}
	\notag \\
	\Leftrightarrow\ \frac{i}{\hbar} \comm{\rho}{\mt{A}(t)} &= -\rho \int\limits_{0}^{\beta} \dd{\lambda} \dot{\mt{A}}(t+\tilde{\lambda}) = -\rho \int\limits_{0}^{\beta} \dd{\lambda} \dot{\mt{A}}(t-i\lambda\hbar),
	\label{eq: Kubo-identity}
\end{align}
%
where the derivation of $\mt{A}$ with respect to $t$ is symbolized with the dot above $\mt{A}$. 
For reasons of lucidity $\tilde{\lambda} = -i\lambda\hbar$ is introduced through the computation.

Now inserting equation \eqref{eq: identity expectation value} and \eqref{eq: Kubo-identity} in the Kubo relaxation function \eqref{eq: Kubo relaxation function} yield the searching form of the Kubo relaxation function, where the right hand side of the following compuation has to be integrated by parts, dedicated with PI.
%
\begin{align}
	\Phi_{\mt{AB}}(t) &= \frac{i}{\hbar} \lim\limits_{s \to 0} \int\limits_{t}^{\infty} \dd{\tau} \expval{\comm{\mt{A}_{\mt{I}}(\tau)}{\mt{B}_{\mt{I}}(0)}}_{0} e^{-s\tau}
	\notag \\
	\overset{\eqref{eq: identity expectation value}}{\Leftrightarrow}\ \Phi_{\mt{AB}}(t) &= \frac{i}{\hbar} \lim\limits_{s \to 0} \int\limits_{t}^{\infty} \dd{\tau} \frac{1}{Z_{0}} \Tr{\comm{\rho_{0}}{\mt{A}_{\mt{I}}(\tau)} \mt{B}_{\mt{I}}(0)} e^{-s\tau}
	\notag \\
	\overset{\eqref{eq: Kubo-identity}}{\Leftrightarrow}\ \Phi_{\mt{AB}}(t) &= -\lim\limits_{s \to 0} \int\limits_{0}^{\beta} \dd{\lambda} \int\limits_{t}^{\infty} \dd{\tau} \expval{\dot{\mt{A}}_{\mt{I}}(\tau-i\lambda\hbar) \mt{B}_{\mt{I}}(0)}_{0} e^{-s\tau}
	\notag \\
	\overset{\mt{PI}}{\Leftrightarrow}\ \Phi_{\mt{AB}}(t) &= -\lim\limits_{s \to 0} \int\limits_{0}^{\beta} \dd{\lambda} \expval{\Bigg[\eval{\mt{A}_{\mt{I}}(\tau-i\lambda\hbar) e^{-s\tau}}_{t}^{\infty} + s \int\limits_{t}^{\infty} \dd{\tau} \dot{\mt{A}}_{\mt{I}}(\tau-i\lambda\hbar) e^{-s\tau} \Bigg] \mt{B}_{\mt{I}}(0)}_{0}
	\notag \\
	\Leftrightarrow\ \Phi_{\mt{AB}}(t) &= \int\limits_{0}^{\beta} \dd{\lambda} \expval{\mt{A}_{\mt{I}}(t-i\lambda\hbar) \mt{B}_{\mt{I}}(0)}_{0} = \int\limits_{0}^{\beta} \dd{\lambda} \expval{\mt{A}_{\mt{I}}(t) \mt{B}_{\mt{I}}(i\lambda\hbar)}_{0}
	\label{eq: Kubo relaxation function 2.0}
\end{align}
%
Later we will see that the scalar product defining at the memory-matrix-formalsim has a similar structure as this form of the kubo relaxation function.
This provide the oppertunity to transform the correlation function out of the language of the memory-matrix-formalism into the Kubo relaxation function, which in turn provide the oppertunity to compute the correlation function pertubativly.
However the should be enough for the fist time.
Later the transformation is discussed in more detail.
%
%
\subsection{Kramer-Kronig-relation}
\label{subsec: Kramer-Kronig-relation}
%
%
All experiences of a human life demonstating that an incident is always bevor the reaction of a system to it. 
In physics this is called causality.
Causality and the condition that the dynamical sysceptibilty $\chi_{\mt{AB}}(t-t')$ is zero for times $t$ smaller than $t'$ are aquivalent assertions.
It's often usefull to work in the frequency space why we want to investigate what causality means in Fourier space.
Consider the Fourier transformation $\chi_{\mt{AB}}(\omega)$ where $\omega$ is replaced by the complex number $\omega'+i\omega''$.
For reasons of simplification the origin of the time axis is set to $t'$.
%
\begin{align}
	\chi_{\mt{AB}}(\omega) = \int\limits_{-\infty}^{\infty} \dd{t} e^{i(\omega'+i\omega'')t} \chi_{\mt{AB}}(t)
	\label{eq: Fourier-transformation}
\end{align}
%
The integral converge if the exponential functions decrease to zero.
Causality in time space yield $t>0$ and because of that $e^{-\omega''t}$ decreases only for $\omega''>0$ to zero.
In summary causality in Fourier space means that the susceptibility is holomorphic in the upper complex plane ($\Im{\omega} = \omega'' > 0$).

Cauchy's integral theorem offers us the oppertunity to express the Fourier transformed susceptibility by a contour integral, where the arbitrary contour $\Gamma$ has to be taken in the upper complex plane or more presicly in the regime where $\chi_{\mt{AB}}(\omega)$ is holomorphic.
%
\begin{align}
	\chi_{\mt{AB}}(\omega) = \frac{1}{2\pi i} \oint\limits_{\Gamma} \dd{\zeta} \frac{\chi_{\mt{AB}}(\zeta)}{\zeta-\omega}
\end{align}
%
Our choice of the contour is some which goes from minus infity to infinity along the real part axis.
Along a semi circle in the upper half plane the contour is closed, see figure \todo{link to figure of contour}.
For reason of convergency the contour along the real part axis is moved in the upper half plane infinitesimal indicated with $i\eta$ where $\eta \to 0$ is implicated.

The contribution of the semi circle vanishs because $\chi_{\mt{AB}}(\omega)$ decreasing very fast for large values of $\omega$ is assumed.
Only a integral along the real part axis survives which can be evaluated by formally writing $\frac{1}{x+i\eta} = \mt{PV}\frac{1}{x}-i\pi\delta(x)$ where PV stands for taking the principal value.
%
\begin{align}
	\chi_{\mt{AB}}(\omega) &= \frac{1}{2\pi i} \int\limits_{-\infty}^{\infty} \dd{\omega'} \frac{\chi_{\mt{AB}}(\omega')}{\omega'-\omega-i\eta} 
	\notag \\
	\Leftrightarrow\ \chi_{\mt{AB}}(\omega) &= \frac{1}{2\pi i} \bigg[
		\mt{PV} \int\limits_{-\infty}^{\infty} \dd{\omega'} \frac{\chi_{\mt{AB}}(\omega')}{\omega'-\omega} 
		+ 
		i\pi \int\limits_{-\infty}^{\infty} \dd{\omega'} \chi_{\mt{AB}}(\omega') \delta(\omega'-\omega)
	\bigg]
	\notag \\
	\Leftrightarrow\ \chi_{\mt{AB}}(\omega) &= -\frac{i}{\pi} \mt{PV} \int\limits_{-\infty}^{\infty} \dd{\omega'} \frac{\Re{\chi_{\mt{AB}}(\omega')} + i\Im{\chi_{\mt{AB}}(\omega')}}{\omega'-\omega} 
	\notag \\
	\Leftrightarrow\ \chi_{\mt{AB}}(\omega) &= \frac{1}{\pi} \mt{PV} \int\limits_{-\infty}^{\infty} \dd{\omega'} \bigg[
		\frac{\Im{\chi_{\mt{AB}}(\omega')}}{\omega'-\omega}
		-i
		\frac{\Re{\chi_{\mt{AB}}(\omega')}}{\omega'-\omega} 
	\bigg]
\end{align}
%
In the second step one right hand side the complex susceptibility is written explicitly by her real and imaginary part.
Nothing keep us from doing this on the left side hand too and compare the real and imaginary parts of both sides respectively.
%
\begin{align}
	\Re{\chi_{\mt{AB}}(\omega)} &= \frac{1}{\pi} \mt{PV} \int\limits_{-\infty}^{\infty} \dd{\omega'} \frac{\Im{\chi_{\mt{AB}}(\omega')}}{\omega'-\omega}
	\\
	\Im{\chi_{\mt{AB}}(\omega)} &= -\frac{1}{\pi} \mt{PV} \int\limits_{-\infty}^{\infty} \dd{\omega'} \frac{\Re{\chi_{\mt{AB}}(\omega')}}{\omega'-\omega}
\end{align}
%
These two relations are called Kramer-Kronig-relation.
They take the real and imaginary part of the a function, here the susceptibility, in a very usefull relation.
In the later computation them are used to compute the Green function on the real axis out off the Green function on the imaginary axis and vice versa.
This is always needed if analytical continuation isn't possible, which is the case considering damping in the Green function.
%
%
\subsection{Spectral representation}
\label{subsec: spectral representation}
%
%
In section \ref{subsec: kubo formula} the dynamical susecptibility $\chi_{\mt{AB}}$ is introduced by deviated the Kubo-formula \eqref{eq: Kubo formula}.
The evolution of a system switching on a pertubation is described by this function.
Now the processes starting because of the pertubation can be classified into two types one the one hand in dissipative prozesses and on the other hand in non-dissipative prozesses.
In the following computation dissipative processes are investigated.
For that a dissipative susceptibility of the form
%
\begin{align}
	\chi''_{\mt{AB}}(t) = \frac{1}{2\hbar} \expval{\comm{\mt{A}(t)}{\mt{B}(0)}}
	\label{eq: dissipative susceptibility}
\end{align}
%
is considered and her Fourier transformation is given by equation \eqref{eq: Fourier-transformation}.
Notice that in the following computation the frequency $\omega$ isn't splitted into real and imaginary part like it's done in equation \eqref{eq: Fourier-transformation}.
Starting our calculation multipling equation \eqref{eq: dynamical susceptibilty} and integrating over time $t$.
%
\begin{align}
	\chi_{\mt{AB}}(t) &= \frac{i}{\hbar} \Theta(t) \expval{\comm{\mt{A}(t)}{\mt{B}(0)}} = 2i \Theta(t) \chi''_{\mt{AB}}(t)
	\notag \\
	\Leftrightarrow\ \chi_{\mt{AB}}(\omega) &= 2i \int\limits_{-\infty}^{\infty} \dd{t} e^{i\omega t} \Theta(t) \chi''_{\mt{AB}}(t)
	\notag \\
	\Leftrightarrow\ \chi_{\mt{AB}}(\omega) &= -\frac{1}{\pi} \lim\limits_{\eta \to 0} \int\limits_{-\infty}^{\infty} \dd{\omega'}
 \frac{1}{\omega' + i\eta} \int\limits_{-\infty}^{\infty} \dd{t} e^{i(\omega-\omega')t} \chi''_{\mt{AB}}(t)
 	\notag \\
	\Leftrightarrow\ \chi_{\mt{AB}}(\omega) &= \frac{1}{\pi} \lim\limits_{\eta \to 0} \int\limits_{-\infty}^{\infty} \dd{\omega'}
 \frac{\chi''_{\mt{AB}}(\omega')}{\omega' - \omega - i\eta} 
 	\notag \\
	\Leftrightarrow\ \chi_{\mt{AB}}(\omega) &= \frac{1}{\pi} \mt{PV} \int\limits_{-\infty}^{\infty} \dd{\omega'}
 \frac{\chi''_{\mt{AB}}(\omega')}{\omega' - \omega} + i \int\limits_{-\infty}^{\infty} \dd{\omega'} \delta(\omega' - \omega) \chi''_{\mt{AB}}(\omega')
 	\notag \\
	\Leftrightarrow\ \chi_{\mt{AB}}(\omega) &= \chi'_{\mt{AB}}(\omega) + i \chi''_{\mt{AB}}(\omega)
	\label{eq: splitting susceptibility into real and imaginary part}
\end{align}
%
where in the second step the following definition of the $\Theta$-function is used.
%
\begin{align}
	\Theta_{\eta}(t) = i \lim\limits_{\eta \to 0} \int\limits_{-\infty}^{\infty} \frac{\dd{\omega'}}{2\pi} \frac{e^{-i\omega't}}{\omega' + i\eta} 
\end{align}
%
In equation \eqref{eq: splitting susceptibility into real and imaginary part} we see that the dynamical susceptibility $\chi_{\mt{AB}}(\omega)$ is seperated into two functions $\chi'_{\mt{AB}}(\omega)$ and $\chi''_{\mt{AB}}(\omega)$, where the latter is the dissipative susceptibility defined at the beginning of this section in \eqref{eq: dissipative susceptibility}.
Equation \eqref{eq: splitting susceptibility into real and imaginary part} is general and for any susceptibility valid.
Assuming the dissipative susceptibility is a real number, than this is also valid for $\chi'_{\mt{AB}}(\omega)$ and the both functions $\chi'_{\mt{AB}}(\omega)$ and $\chi''_{\mt{AB}}(\omega)$ are real and imaginary part of $\chi_{\mt{AB}}(\omega)$, respectivily.
%
%
%
\section{Deviation of the Memory-Matrix-Formalism}
\label{sec: deviation of the memory-matrix-formalism}
%
%
%
After this short reminder of the linear response theorie and the investigation of the dynamical susceptibility the groundwork for the deviation of the memory-matrix-formalism is done and we want to go back.
This chapter started by splitting a dynamical observable into two parts, a secular and a non-secular one.
The systematical evolution of observables is determined by the secular part.
Looking at the system after a pertubation is switched off for a long time this secular part is depended the evolution.
Furthermore all processes with a short lifetime or small quantity compared with the linear term in pertubation series are summerized in the non-secular part.
This result is the starting point to a simple geometrical interpretation in a vector space, which we want define in the following.

Therefore the mathematical framework in quantum mechanics has to be clear, why a short review based on \cite{Audretsch} is given in the following.
A $d$-dimensional Hilbert-space is mormaly the mathematical working area in quantum mechanics.
This vector space is linear, complex and has a defined scalar product.
The vectors $\ket{\phi}$, usually denoted in the Dirac-notation, are identified with all possible states for the system.
Because the man is always interested in observables, linear operators are defined in the Hilbert-space where the eigenvalues of them conform to the observables.
Defining the dyad product $\sum_{i} \dyad{i}{i}$ it's not hard to see that any linear operator occupies a dyad decomposition
%
\begin{align}
	\mt{A} = \sum\limits_{i,j} \dyad{i}{i} \mt{A} \dyad{j}{j} = \sum\limits_{i,j} \mt{A}_{ij} \ket{i} \bra{j},
	\label{eq: dyad product}
\end{align}
%
where $\mt{A}_{ij} := \bra{i}\mt{A}\ket{j}$ is a matrix element of the linear operator.
The dyad product of an operator is now used to introduce a new vector space of all linear operators acting on the $d$-dimensional Hilbert-space which is called the Liouville-space $\mathbb{L}$ or operator space.

The Liouville-space is linear and complex vector space equally to the Hilbert-space.
The difference between both are the vectors or elements living in the space.
In the Liouville-space the vectors are linear operators $\mt{A}, \mt{B}, \dots$ which are acting on some Hilbert-space.
In other words this means that the dyad decomposition of an vector in the $d$-dimensional Hilbert-space is the new vector in the Liouville-space.
So some vector in the Liouville-space is notated as
%
\begin{align}
	\oket{\mt{A}} := \sum\limits_{ij}^{d} \mt{A}_{ij} \oket{\dyad{i}{j}}
	\label{eq: definition of a vector in Liouville-space}
\end{align}
%
Similiarly to the quantum mechanic the Dirac notation is used with the difference that round brackets are used instead of angle brackets to distinghush both spaces.
Out of the definition \eqref{eq: definition of a vector in Liouville-space} it's clear, that the basis in the Liouville-space is build by the $d^{2}$ dyads of the Hilbert-space .
The dimension of the Liouville-space is therefore $d^{2}$.
Equally to a Hilbert-space there are many other oppertunities to choose the basis in the Liouville space $\mathbb{L}$, but the defintion in \eqref{eq: definition of a vector in Liouville-space} should be the one we are working with.

In the following the basis of our Liouville space is denoted with $\{\toket{\mt{A}_{i}}\}$ where $i = 1,2,3,\dots,n$ and $\mt{A}_{i}$ is an operator.
The corresponding basis of the dual space is given by $\{\tobra{\mt{A}_{i}}\}$, similarily to the Hilbert space.
The last needed element of our Liouville space is a scalar product which fullfills the three condictions
%
\begin{enumerate}
	\item $\begin{aligned} \obraket{\mt{A}_{i}}{\mt{A}_{j}} = \obraket{\mt{A}_{j}}{\mt{A}_{i}}^{*} \end{aligned}$
	\item $\begin{aligned} \obraket{\mt{A}_{i}}{\mt{B}} = c_{1} \obraket{\mt{A}_{i}}{\mt{A}_{j}} + c_{2} \obraket{\mt{A}_{i}}{\mt{A}_{k}} \qq{with} \mt{B} = c_{1} \mt{A_{j}} + c_{2} \mt{A_{k}} \qq{and} c_{1}, c_{2} \in \mathbb{C} \end{aligned}$
	\item $\begin{aligned} \obraket{\mt{A}_{i}}{\mt{A}_{i}}\geq 0 \qq{, where equallity is fulfilled if} \mt{A}_{i} = 0. \end{aligned}$
\end{enumerate}
%
Beside these the choice of the scalar product is arbitrary.
For the moment let us choose 
%
\begin{align}
	\obraket{\mt{A}_{i}(t)}{\mt{A}_{j}(t')} = \frac{1}{\beta} \int\limits_{0}^{\beta} \dd{\lambda} \expval{\mt{A}_{i}^{\dag}(t) \mt{A}_{j}(t'+i\lambda\hbar)}
	\label{eq: scalar product Liouville space}
\end{align}
%
as our scalar product, where the normal time evolution of an operator \linebreak$\mt{A}_{i}(t) = e^{i\mt{H}t/\hbar} \mt{A}_{i}(0) e^{-i\mt{H}t/\hbar}$ is valid, so that $\mt{A}_{i}(i\lambda\hbar) = e^{-\lambda\mt{H}} \mt{A}_{i}(0) e^{\lambda\mt{H}}$ can be used.
A more detailed discussion of the choice of the sclar product is given at the end of this chapter in \todo{link to the section where the scalar product is motivated}.
Now we have to proof if the condictions are fullfilled by the choice of our scalar product.

Let's get started with the second one because it's easily shown transforming the expactation value into the trace representation and then using the properties of the trace. \todo{maybe the computation isn't needed here}
%
\begin{align}
	\obraket{\mt{A}_{i}(t)}{\mt{B}(t')} &= \frac{1}{\beta} \int\limits_{0}^{\beta} \dd{\lambda} \frac{1}{Z} \Tr{\rho \mt{A}_{i}^{\dag}(t) \Big[c_{1} \mt{A}_{j}(t'+i\lambda\hbar) + c_{2} \mt{A}_{k}(t'+i\lambda\hbar)\Big]}
	\notag \\
	\Leftrightarrow\ \obraket{\mt{A}_{i}(t)}{\mt{B}(t')} &= c_{1} \obraket{\mt{A}_{i}^{\dag}(t)}{\mt{A}_{j}(t')} + c_{2} \obraket{\mt{A}_{i}^{\dag}(t)}{\mt{A}_{k}(t')}
\end{align}
%
The first and third condition can be shown by transforming the scalar product in the spectral representation.
Therefore the trace is writen explicitly as a sum over all states and the unity operator written as a sum over all projection operators are inserted between both operators $\mt{A}_{i}$ and $\mt{A}_{j}$. 
%
\begin{align}
	\obraket{\mt{A}_{i}(t)}{\mt{A}_{j}(t')} &= \frac{1}{\beta \cdot Z} \int\limits_{0}^{\beta} \dd{\lambda} \sum\limits_{n,m} \bra{n} e^{-\beta \mt{H}} \mt{A}_{i}^{\dag}(t) \ket{m} \bra{m} e^{-\lambda \mt{H}} \mt{A}_{j}(t') e^{\lambda \mt{H}} \ket{n}
	\notag \\
	\Leftrightarrow\ \obraket{\mt{A}_{i}(t)}{\mt{A}_{j}(t')} &= \frac{1}{\beta \cdot Z} \sum\limits_{n,m} \mel{n}{\mt{A}_{i}^{\dag}(t)}{m} \mel{m}{\mt{A}_{j}(t')}{n} e^{-\beta E_{n}} \int\limits_{0}^{\beta} \dd{\lambda} e^{\lambda (E_{n}-E_{m})} 
	\notag \\
	\Leftrightarrow\ \obraket{\mt{A}_{i}(t)}{\mt{A}_{j}(t')} &= \frac{1}{\beta \cdot Z} \sum\limits_{n,m} \mel{n}{\mt{A}_{i}^{\dag}(t)}{m} \mel{m}{\mt{A}_{j}(t')}{n}  \frac{e^{-\beta E_{m}} - e^{-\beta E_{n}}}{E_{n}-E_{m}}
	\label{eq: expactation value in spectral representation}
\end{align}
%
The complex conjugated of the expectation value in the Liouville space is considered and using $\mel{n}{\mt{A}_{j}^{\dag}(t)}{m}^{*} = \mel{m}{\mt{A}_{j}(t)}{n}$ let us find instantly the first condition.
Notice that on the right hand side of equation \eqref{eq: expactation value in spectral representation} only the expactation values are complex numbers.
For them the complex conjugation yields
%
\begin{align}
	\Big(\mel{n}{\mt{A}_{i}^{\dag}(t)}{m} \mel{m}{\mt{A}_{j}(t')}{n}\Big)^{*} = \mel{n}{\mt{A}_{j}^{\dag}(t')}{m} \mel{m}{\mt{A}_{i}(t)}{n}
\end{align}
%
and inserting back $\tobraket{\mt{A}_{i}(t)}{\mt{A}_{j}(t')}^{*}$ is exactly the same as \eqref{eq: expactation value in spectral representation}.
Proofing the third condition it has to be set $\mt{A}_{j}(t') = \mt{A}_{i}(t)$ in equation \eqref{eq: expactation value in spectral representation}, which one the right hand side results in 
%
\begin{align}
	\obraket{\mt{A}_{i}(t)}{\mt{A}_{i}(t)} &= \frac{1}{\beta \cdot Z} \sum\limits_{n,m} \big\vert\mel{m}{\mt{A}_{i}(t)}{n}\big\vert^{2} \frac{e^{-\beta E_{m}} - e^{-\beta E_{n}}}{E_{n}-E_{m}}.
\end{align}
%
It's clear that the squared expactation value is always non-negative.
The friction is positive too, which is easily seen by proofing the two cases $E_{n} > E_{m}$ and $E_{n} < E_{m}$.
Therefore the expactation value $\obraket{\mt{A}_{i}(t)}{\mt{A}_{i}(t)} \geq 0$ and equallity is only possible if $\mt{A}_{i} = 0$.
All three conditions are well proofed and the chosen scalar product is really one's.
At this point all the mathematical ground work is done, we know how the vectors in the Liouville space looks like and we have a well defined scalar product.

The goal of every physical theory is to describe the measurment results.
Typically in statistical or quantum mechanics this is done by correlations functions.
So our goal is it now to find a useable expression for correlation functions in our new Liouville space.
The natural starting point describing the time evolution of an operator $\mt{A}_{i}$ is in quantum mechanics the Heisenberg equation of motion
%
\begin{align}
	\dv{t} \mt{A}_{i}(t) = \dot{\mt{A}_{i}}(t) = \frac{i}{\hbar} \comm{\mt{H}}{\mt{A}_{i}(t)} = i \mt{L} \mt{A}_{i}(t)
	\label{eq: Heisenberg equation of motion}
\end{align}
%
where the operators are in the Heisenberg representation and the Hermitian Liouville operator defined by his action on an operator ${\mt{L} = \hbar^{-1} \comm{\mt{H}}{\mt{\bullet}}}$ is introduced.
The formal solution of equation \eqref{eq: Heisenberg equation of motion} is
%
\begin{align}
	\mt{A}_{i}(t) = e^{it\mt{L}} \mt{A}_{i}(0) = e^{it\mt{H}/\hbar} \mt{A}_{i}(0) e^{-it\mt{H}/\hbar}.
\end{align}
%
In the second step only the definition of the Liouville operator and some algebraic transformations are used. \todo{Have to convince me that it is really so easy.}
In this notation it is more clearly that the time evolution of an operator is given by the Liouville operator.
The same result is obtained in the Liouville space if the Liouville operator is acting on the basis vectors.
This isn't really surprisingly because only the dyad product has to be insert in equation \eqref{eq: Heisenberg equation of motion}, which results in
%
\begin{align}
	\oket{\dot{\mt{A}}_{i}(t)} = \frac{i}{\hbar} \oket{\comm{\mt{H}}{\mt{A}_{i}(t)}} = i \mt{L} \oket{\mt{A}_{i}(t)}
\end{align}
%
for the equation of motion in the Liouville space and there fromal solution
%
\begin{align}
	\oket{\mt{A}_{i}(t)} = e^{it\mt{L}} \oket{\mt{A}_{i}(0)}.
	\label{eq: formal solution of EM in L}
\end{align}
%
Beside the Liouville operator one more operator has to be introduced for the deviation of the correlation function, called the projection operator.
Therefore let us define a set of operators $\{\mt{C}_{i}\}$, where the choice of these operators are different depending on the investigated system and correlation function.
In the later computation of a certain problem the choice of the operators is discussed in more detail.
For the moment it's sufficient to know that the set of operators exists.
Directly follwing out the definition of the projection operator in quantum mechanics the projection operator in the Liouville space looks like
%
\begin{align}
	\mt{P} = \sum\limits_{i,j} \oket{\mt{C}_{i}(0)} \obraket{\mt{C}_{i}(0)}{\mt{C}_{j}(0)}^{-1} \obra{\mt{C}_{j}(0)}.
	\label{eq: projection operator}
\end{align}
%
The action of P on some vector $\oket{\mt{A}(t)}$ in the Liouville space yields the parallel components to the chosen operators $\mt{C}_{i}$, which is the projection from $\oket{\mt{A}(t)}$ at the vector subspace spanned by $\mt{C}_{i}$.
The corresponding vertical component of $\oket{\mt{A}(t)}$ with respect to the operators $\mt{C}_{i}$ is given by $\mt{Q} = 1- \mt{P}$, which is the projection out of the vector subspace.
Naturally the projection operator is fullfilled the two properties $\mt{P}^{2} = \mt{P}$ and $\mt{PQ} = \mt{QP} = 0$ of a projection operator, which follows immediately from the definition of $\mt{P}$.

After the deviation of the time evolution of an operator and the projection operator in Liouville space the correlation function can be defined as
%
\begin{align}
	\mathcal{C}_{ij}(t) = \obraket{\mt{A}_{i}(t)}{\mt{A}_{j}(0)} \overset{\eqref{eq: scalar product Liouville space}}{=} \frac{1}{\beta} \int\limits_{0}^{\beta} \dd{\lambda} \expval{\mt{A}_{i}(t) \mt{A}_{j}(i\lambda\hbar)}
	\label{eq: correlation function Liouville space}
\end{align}
%
where in the last step the definition of the scalar product is only inserted.
Comparing equation \eqref{eq: correlation function Liouville space} with \eqref{eq: Kubo relaxation function 2.0} our choice of the correlation function is more clear.
The defined correlation function is proportional to the Kubo relaxation function, which how we learned in section \ref{subsec: Kubo relaxation function} describes the system's reaction on a switched off pertubation. \todo{write more to the goal of the memory matrix formalsim}
For $t=0$ the correlation function is also proportional to the Fourier transformated susebtibility
%
\begin{align}
	\mathcal{C}_{ij}(t = 0) = \frac{1}{\beta} \Phi_{ij}(t = 0) = \chi_{ij}(\omega = 0),
	\label{eq: relation between C, Phi and chi}
\end{align}
%
which directly results from equation \eqref{eq: relation 2 between Phi and chi}.
Equation \eqref{eq: formal solution of EM in L} is used to bring the time evolution of the correlation function in more suitable expression
%
\begin{align}
	\mathcal{C}_{ij}(t) = \obraket{\mt{A}_{i}(0)}{\mt{A}_{j}(-t)} = \obra{\mt{A}_{i}(0)} e^{-it\mt{L}} \oket{\mt{A}_{j}(0)},
\end{align}
%
which opens the possibility for using the Laplace transformation.
Instead of the definition in equation \eqref{eq: Laplace transformation real axis} here a form of the Laplace transformation is used where $s$ is substituted by $-i\omega$ which is nothing else a rotation of the definition regime by $\frac{\pi}{2}$.
Multipling the last equation with $e^{i\omega t}$ and intgrate from zero to infinty with resprct to $t$ yields
%
\begin{align}
	\mathcal{C}_{ij}(\omega) = \obra{\mt{A}_{i}} \int\limits_{0}^{\infty} \dd{t} e^{i(\omega-\mt{L})t} \oket{\mt{A}_{j}} = \obra{\mt{A}_{i}} \frac{i}{\omega - \mt{L}} \oket{\mt{A}_{j}},
\end{align}
%
where for reasons of clarity and comprehensibility from now on the argument $t=0$ isn't anymore written at the basis vectors.
Now the relation $\mt{L} = \mt{LQ} + \mt{LP}$ which follows immediatly by using the definition of P and Q and the identity $ (\mt{X} + \mt{Y})^{-1} = \mt{X}^{-1} - \mt{X}^{-1} \mt{Y} (\mt{X} + \mt{Y})^{-1}$ is used to simplify the correlation function, where $\mt{X} = \omega - \mt{LQ}$ and $\mt{Y} = -\mt{LP}$.
%
\begin{align}
	\mathcal{C}_{ij}(\omega) &= \obra{\mt{A}_{i}} \frac{i}{\omega - \mt{LQ} - \mt{LP}} \oket{\mt{A}_{j}}
	\notag \\
	\Leftrightarrow\ \mathcal{C}_{ij}(\omega) &= \obra{\mt{A}_{i}} \frac{i}{\omega - \mt{LQ}} \oket{\mt{A}_{j}} + \obra{\mt{A}_{i}} \frac{1}{\omega - \mt{LQ}} \mt{LP} \frac{i}{\omega - \mt{L}} \oket{\mt{A}_{j}}
\end{align}
%
The both terms on the right hand side are considered seperatly starting with the first one.
The fraction can be written as the geometric series assuming $\frac{\mt{LQ}}{\omega} < 1$, which means that the pertubation is small compared to other quantities in the system. \todo{Ask J\"org if this explanation is correct}
%
\begin{align}
	\frac{i}{\omega - \mt{LQ}} = \frac{i}{\omega} \bigg[1 + \frac{\mt{LQ}}{\omega} + \Big(\frac{\mt{LQ}}{\omega}\Big)^{2} + \dots \bigg]
\end{align}
%
Each term of the series in the squard brackets acting on the operator $\oket{\mt{A}_{j}}$.
Remember this is the operator at time $t=0$, which means that no vertical component exists and therefore $\mt{Q}\oket{\mt{A}_{j}} = 0$.
Every term except the first one conatins an operator $Q$, so the first term of the correlation function yields
%
\begin{align}
	\obra{\mt{A}_{i}} \frac{i}{\omega - \mt{LQ}} \oket{\mt{A}_{j}} = \frac{i}{\omega} \obraket{\mt{A}_{i}}{\mt{A}_{j}} = \frac{i}{\omega} \mathcal{C}_{ij}(t=0).
\end{align}
%
At the second term only the back term is considered.
Here only the explicit expression of the propagator is inserted, which yields the definition of the Laplace transformed correlation function.
%
\begin{align}
	\mt{P} \frac{i}{\omega - \mt{L}} \oket{\mt{A}_{j}} = \sum\limits_{k,l} \oket{\mt{C}_{k}} \obraket{\mt{C}_{k}}{\mt{C}_{l}}^{-1} \obra{\mt{C}_{l}} \frac{i}{\omega - \mt{L}} \oket{\mt{A}_{j}} = \sum\limits_{k,l} \oket{\mt{C}_{k}} \mathcal{C}_{kl}^{-1}(0) \mathcal{C}_{lj}(\omega)
\end{align}
%
Inserting back both simplifications the correlation function is get the formal expression
%
\begin{align}
	\mathcal{C}_{ij}(\omega) = \frac{i}{\omega} \mathcal{C}_{ij}(t=0) + \sum\limits_{k,l} \obra{\mt{A}_{i}} \frac{1}{\omega - \mt{LQ}} \mt{L} \oket{\mt{C}_{k}} \mathcal{C}_{kl}^{-1}(0) \mathcal{C}_{lj}(\omega).
\end{align}
%
























































