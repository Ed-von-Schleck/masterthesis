%
%
\chapter{Memory-Matrix-Formalsim}
\label{ch: memory-matrix-formalism}
%
%
\section{Motivation}
\label{sec: motivation}
%
%
A physicist is always interested in the beaviour and time evolution of the observables of the investigates system.
In the middle of the last century many physicists worked on the understanding and mathematical description of one physical process, the Brownian motion.
On stochasical theory of these certain physical process is based on the Langevin equation
%
\begin{align}
	\pdv{t} \mt{A}(t) -\mt{F}_{\mt{ex}}(x,t) + \gamma \cdot \mt{A}(t) = f(t),
	\label{eq: Langevin equation}
\end{align}
%
where $\mt{A}(t)$ is some dynamical observable and $f(t)$ is a random force like white noise for example.
The origin of the second term on the left hand side is some external force result from a coupling between $\mt{A}(t)$ and some external potential.
The third term on the left hand side is a damping or friction term.
Now let us assume it's possible to seperate equation \eqref{eq: Langevin equation} into two parts.
The first part, called $f_{1}$, is a functional of the dynamical observable $\mt{A}(t')$, where $t_{0} \leq t' \leq t$, so that this part is depending on the history of A.
The second part $f_{2}$ should be depending on all other degrees of freedom.
Now $f_{1}$ is expanded up to the linear order and all terms of higher order and the part $f_{2}$ are summerized to the quantity $F(t)$.
The result is a linearized form of the Langevin equation
%
\begin{align}
	\pdv{t} \mt{A}(t) = \int\limits_{t_{0}}^{t} \dd{t'} \mathcal{C}(t-t') \mt{A}(t') + F(t),
	\label{eq: linearized Langevin equation}
\end{align}
%
where $\mathcal{C}$ is a correlation function and $\mt{A}(t')$ is the deviation of the invariant part of the Hamiltonian.
For large time scales the deviation should be vanish, so the time-integral over $\mt{A}(t')$ should be become zero.
For simplification the origin of the time axis is moved to $t_{0}$.
In general the Laplace transformation of a function is given by
%
\begin{align}
	\mathcal{L}\big\{\mt{A}(t)\big\} = \mt{A}(s) = \int\limits_{0}^{\infty} \dd{t} \mt{A}(t) e^{-st}.
	\label{eq: Laplace transformation real axis}
\end{align}
%
Using the Laplace transformation equation \eqref{eq: linearized Langevin equation} becomes a algebratic equation of motion.
The solution of this equation is 
%
\begin{align}
	\mt{A}(t) = \Xi(t) \cdot \mt{A}(0) + \mt{A}'(t) \hspace{1cm} \mt{with} \hspace{1cm} \mt{A}'(t) = \int\limits_{0}^{t} \dd{t'} \Xi(t-t') F(t'),
	\label{eq: splitted observable}
\end{align}
%
where the function $\Xi(t)$ is defined by the Laplace transformation of $\Xi(s) = [s-\mathcal{C}(s)]^{-1}$ and $\mathcal{C}(s)$ is the Laplace transformtion of the correlation function $\mathcal{C}(t)$.
The main result of equation \eqref{eq: splitted observable} and the motivation for the following introduced memory-matrix-formalism is the splitting of the dynamical observable $\mt{A}(t)$ into two parts.

For the first term on the right hand side the only time-dependence is adverted through the correlation function $\mathcal{C}$, which is clear regarding the definition of $\Xi$.
This term included the linear contributions of $\mt{A}(t)$ by construction.
These ones are the mostly important contributions to the time evolutaion of an observable, because they are secular.
In contrast the second term on the right hand side is the convolution between the function $\Xi(t-t')$ and the function $\mt{F}(t')$.
The latter summerize all the non-linear effects, fluctuations and intital transient processes, which are all effects with a small lifetimes in contrast with the secular effects.
Therefore these effects shouldn't have large influences on the time evolution of an observable, always large time scales in mind.

Beside the physical interpretation a simple geometrical and mathematical one is very usefull.
Let us assume a vector space ana the observable should be a vector in this vector space.
Then the secular term is a projection on the A-axis and the non-secular term is aquivalent to a vector perpendicular to the A-axis.
The memory-matrix-formalism take up this simple interpretation of equation \eqref{eq: splitted observable} and put it in a general and exact form, so that it can be used classicaly and quantum mechanicaly.
%
%
\section{Linear Response Theory}
\label{sec: linear response theory}
%
%
Before the derivation of the memory-matrix-formalism can be started some ground work is to do.
This section begins with a short reminder of the kubo formula. %also known as linear response theory.
After that the Kubo relaxation function are introduced and some important relations between there and the retarded susceptibility $\chi$ are derivated.
In the last section finally the splitting of $\chi$ in a real and an imaginary part are dicussed.
%
%
\subsection{Kubo formula}
\label{subsec: kubo formula}
%
%
Consider a system in equilibrium represented by the Hamiltonian $H_{0}$.
At an arbitrary time $t'$ a pertubation is switched on, where the pertubation is given by the Hamiltonian $H_{1} = - B \cdot F(t)$, so that $H(t) = H_{0} + H_{1}$ is the full Hamiltonian.
Thereby $B$ is an operator by which the pertubation is coupled on the system and $F(t)$ is a function determining the time evolution of the pertubation.
It is assumed that $F(t) = 0$ for $t<t'$ so that the system is in thermal equilibrium for all these times.

The physical interest is existed in the question how does an observable $\mt{A}$ react on the pertubation switched on at $t'$.
The answer is given by the thermodynamical expectation value of the operator corresponding to the observable $\mt{A}$
%
\begin{align}
	\expval{\mt{A}}(t) := \Tr{\rho_{\mt{S}}(t) \mt{A}_{\mt{S}}} = \Tr{\rho_{\mt{I}}(t) \mt{A}_{\mt{I}}},
	\label{eq: thermodynamical expectation value}
\end{align}
%
where the label S and I stand for the Schrödinger and Interaction picture, respectivily.
The equality of the expectation value in the different regarded pictures is shown by the invariance of the trace under cycle permutation.
The transformation into the interaction picture is very usefull what we will see after the next step below.
In quantum mechnics the time evolution of the density operator is determined by the von Neumann-equation.
%
\begin{align}
	\dv{t} \rho_{\mt{S}}(t) = -\frac{i}{\hbar} \comm{H(t)}{\rho_{\mt{S}}(t)} \hspace{0.5cm}\Leftrightarrow\hspace{0.5cm} \dv{t} \rho_{\mt{I}}(t) = -\frac{i}{\hbar} \comm{H_{1}}{\rho_{\mt{I}}(t)}
	\label{eq: von Neumann-equation}
\end{align}
%
The equation is also transformed into the interaction picture, which doesn't change the structure itself but the density operator deponds only on the Hamiltonian $H_{1}$ now.
Integrating and using the boundary condition that the system is in thermal equilibrium at $t \to -\infty$ equation \eqref{eq: von Neumann-equation} is resulted in a integrable equation for the density operator.
%
\begin{align}
	\rho_{\mt{I}}(t) = \rho_{0} + \frac{i}{\hbar} \int\limits_{-\infty}^{t} \dd{t'} \comm{B_{\mt{I}}(t')}{\rho_{\mt{I}}(t')}F(t')
	\label{eq: integrable form of von Neumann-equation}
\end{align}
%
Jet it is clear why the interection picture is used.
The integrand depends on the Hamiltonian of the pertubation only in linear order which is a perfect starting point for a iterativ solution procedure.
Starting with the zeroth order the density operator is trivially the density operator at thermical equilibrium.
Inserting the zeroth order on the right hand side of equation \eqref{eq: integrable form of von Neumann-equation} yield the first order of the density operator, a.\,s.\o.
In linear response theory the iteration is cutted off after the first order.
Inserting this in equation \eqref{eq: thermodynamical expectation value} and defining the dynamical susceptibility 
%
\begin{align}
	\chi_{\mt{AB}}(t-t') = \frac{i}{\hbar} \Theta(t-t') \expval{\comm{A_{\mt{I}}(t-t')}{B_{\mt{I}}(0)}}_{H_{0}}
	\label{eq: dynamical susceptibilty}
\end{align}
%
yield the Kubo formula
%
\begin{align}
	\delta\expval{\mt{A}(t)} := \expval{\mt{A}}(t) - \expval{\mt{A}(t)}_{H_{0}} \approx \int\limits_{-\infty}^{\infty} \dd{t'} \chi_{\mt{AB}}(t-t') F(t'),
	\label{eq: Kubo formula}
\end{align}
%
where the label $H_{0}$ means that the expactation value is taken with respect to the unpertubated Hamiltonian.
We see that the deviation of the observable A caused by the pertubation is given by the convolution of the dynamical suszeptibilty $\chi_{\mt{AB}}(t-t')$ and the time evolution function $F(t)$.\todo{noch sch\"oner schreiben}
%
%
\subsection{Kubo relaxation function}
\label{subsec: Kubo relaxation function}
%
%
After a general equation for the deviation of an observable A from the equilibrium value was established, we want to investigate a certain kind of pertubation.
Let us assume $F(t) = \Theta(-t) \cdot F \cdot e^{-s\tau}$ the time evolution function of a pertubation, which is switched on adiabatically at $t=-\infty$ and switched off at $t=0$.
Inserting this in equation \eqref{eq: Kubo formula} and substituting $\tau = t-t'$ yield $\delta\expval{\mt{A}(t)} = \Phi_{AB}(t) \cdot F e^{st}$ with the Kubo relaxation function
%
\begin{align}
	\Phi_{\mt{AB}}(t) = \frac{i}{\hbar} \lim\limits_{s \to 0} \int\limits_{t}^{\infty} \dd{\tau} \expval{\comm{\mt{A}_{\mt{I}}(\tau)}{\mt{B}_{\mt{I}}(0)}}_{0} e^{-s\tau}.
	\label{eq: Kubo relaxation function}
\end{align}
%
The arising $\Theta$-distributions determine the lower limit of the intergal to $t$.
For a more detailed derivation of the Kubo relaxation function see \cite{Schwabl} or \cite{Schwabl2}.
It's not really surprisingly that the Kubo relaxation function and the dynamical susceptibility are closely connected, because the first is derivated out of the latter one.
However there exist three very important relations between them both, which are 
%
\begin{enumerate}
	\item $\begin{aligned}[t] \chi_{\mt{AB}}(t) = -\Theta(t) \dv{t} \Phi_{\mt{AB}}(t) \label{eq: relation 1 between Phi and chi} \end{aligned}$
	\item $\begin{aligned} \Phi_{\mt{AB}}(t = 0) = \chi_{\mt{AB}}(\omega = 0) \label{eq: relation 2 between Phi and chi} \end{aligned}$
	\item $\begin{aligned} \Phi_{\mt{AB}}(\omega) = \frac{1}{i\omega}\big[\chi_{\mt{AB}}(\omega) - \chi_{\mt{AB}}(\omega = 0)\big]. \label{eq: relation 3 between Phi and chi} \end{aligned}$
\end{enumerate}
%
The evidence of these tree relations are shown in the appendix \ref{app: properties of the Kubo relaxation function}.
For the later deviation of the memory-matrix-formalism it's more usefull to write the Kubo relaxation function in another, not so intuitivly form.
The goal of the rewriting is to get the expectation value in a form with no commutator and to do this two identities are needed.
The first one is
%
\begin{align}
	\expval{\comm{\mt{A}(t)}{\mt{B}(t')}} &= \frac{1}{Z} \Tr{\comm{\rho}{\mt{A}(t)} \mt{B}(t')},
	\label{eq: identity expectation value}
\end{align}
%
where the invariance of the expactation value with respect to cycling permutation is used.
The second one is the Kubo-identity.
Thereby the main idea is to used the analogy of the exponential functions to the time evolution of an operator.
%
\begin{align}
	i \comm{\rho}{\mt{A}(t)} &= i \Big[\rho \mt{A}(t) - \mt{A}(t) \rho\Big]
	\notag \\
	\Leftrightarrow\ i \comm{\rho}{\mt{A}(t)} &= i \Big[\rho \mt{A}(t) - e^{-\beta H} e^{\beta \mt{H}} \mt{A}(t) e^{-\beta \mt{H}}\Big]
	\notag \\
	\Leftrightarrow\ i \comm{\rho}{\mt{A}(t)} &= -i \rho \int\limits_{0}^{\beta} \dd{\lambda} \dv{\lambda} e^{\lambda \mt{H}} \mt{A}(t) e^{-\lambda \mt{H}}
	\notag \\
	\Leftrightarrow\ i \comm{\rho}{\mt{A}(t)} &= -i \rho \int\limits_{0}^{\beta} \dd{\lambda} \bigg[\mt{H} e^{i\tilde{\lambda} \mt{H}/\hbar} \mt{A}(t) e^{-i\tilde{\lambda} \mt{H}/\hbar} - e^{i\tilde{\lambda} \mt{H}/\hbar} \mt{A}(t) e^{-i\tilde{\lambda} \mt{H}/\hbar} \mt{H}\bigg]
	\notag \\
	\Leftrightarrow\ i \comm{\rho}{\mt{A}(t)} &= -i \rho \int\limits_{0}^{\beta} \dd{\lambda} \comm{\mt{H}}{\mt{A}(t+\tilde{\lambda})}
	\notag \\
	\Leftrightarrow\ \frac{i}{\hbar} \comm{\rho}{\mt{A}(t)} &= -\rho \int\limits_{0}^{\beta} \dd{\lambda} \dot{\mt{A}}(t+\tilde{\lambda}) = -\rho \int\limits_{0}^{\beta} \dd{\lambda} \dot{\mt{A}}(t-i\lambda\hbar),
	\label{eq: Kubo-identity}
\end{align}
%
where the derivation of $\mt{A}$ with respect to $t$ is symbolized with the dot above $\mt{A}$. 
For reasons of lucidity $\tilde{\lambda} = -i\lambda\hbar$ is introduced through the computation.

Now inserting equation \eqref{eq: identity expectation value} and \eqref{eq: Kubo-identity} in the Kubo relaxation function \eqref{eq: Kubo relaxation function} yield the searching form of the Kubo relaxation function, where the right hand side of the following compuation has to be integrated by parts, dedicated with PI.
%
\begin{align}
	\Phi_{\mt{AB}}(t) &= \frac{i}{\hbar} \lim\limits_{s \to 0} \int\limits_{t}^{\infty} \dd{\tau} \expval{\comm{\mt{A}_{\mt{I}}(\tau)}{\mt{B}_{\mt{I}}(0)}}_{0} e^{-s\tau}
	\notag \\
	\overset{\eqref{eq: identity expectation value}}{\Leftrightarrow}\ \Phi_{\mt{AB}}(t) &= \frac{i}{\hbar} \lim\limits_{s \to 0} \int\limits_{t}^{\infty} \dd{\tau} \frac{1}{Z_{0}} \Tr{\comm{\rho_{0}}{\mt{A}_{\mt{I}}(\tau)} \mt{B}_{\mt{I}}(0)} e^{-s\tau}
	\notag \\
	\overset{\eqref{eq: Kubo-identity}}{\Leftrightarrow}\ \Phi_{\mt{AB}}(t) &= -\lim\limits_{s \to 0} \int\limits_{0}^{\beta} \dd{\lambda} \int\limits_{t}^{\infty} \dd{\tau} \expval{\dot{\mt{A}}_{\mt{I}}(\tau-i\lambda\hbar) \mt{B}_{\mt{I}}(0)}_{0} e^{-s\tau}
	\notag \\
	\overset{\mt{PI}}{\Leftrightarrow}\ \Phi_{\mt{AB}}(t) &= -\lim\limits_{s \to 0} \int\limits_{0}^{\beta} \dd{\lambda} \expval{\Bigg[\eval{\mt{A}_{\mt{I}}(\tau-i\lambda\hbar) e^{-s\tau}}_{t}^{\infty} + s \int\limits_{t}^{\infty} \dd{\tau} \dot{\mt{A}}_{\mt{I}}(\tau-i\lambda\hbar) e^{-s\tau} \Bigg] \mt{B}_{\mt{I}}(0)}_{0}
	\notag \\
	\Leftrightarrow\ \Phi_{\mt{AB}}(t) &= \int\limits_{0}^{\beta} \dd{\lambda} \expval{\mt{A}_{\mt{I}}(t-i\lambda\hbar) \mt{B}_{\mt{I}}(0)}_{0} = \int\limits_{0}^{\beta} \dd{\lambda} \expval{\mt{A}_{\mt{I}}(t) \mt{B}_{\mt{I}}(i\lambda\hbar)}_{0}
\end{align}
%
Later we will see that the scalar product defining at the memory-matrix-formalsim has a similar structure as this form of the kubo relaxation function.
This provide the oppertunity to transform the correlation function out of the language of the memory-matrix-formalism into the Kubo relaxation function, which in turn provide the oppertunity to compute the correlation function pertubativly.
However the should be enough for the fist time.
Later the transformation is discussed in more detail.
%
%
\subsection{Kramer-Kronig-relation}
%
%
All experiences of a human life demonstating that an incident is always bevor the reaction of a system to it. 
In physics this is called causality.
Causality and the condition that the dynamical sysceptibilty $\chi_{\mt{AB}}(t-t')$ is zero for times $t$ smaller than $t'$ are aquivalent assertions.
It's often usefull to work in the frequency space why we want to investigate what causality means in Fourier space.
Consider the Fourier transformation $\chi_{\mt{AB}}(\omega)$ where $\omega$ is replaced by the complex number $\omega'+i\omega''$.
For reasons of simplification the origin of the time axis is set to $t'$.
%
\begin{align}
	\chi_{\mt{AB}}(\omega) = \int\limits_{-\infty}^{\infty} \dd{t} e^{i(\omega'+i\omega'')t} \chi_{\mt{AB}}(t)
\end{align}
%
The integral converge if the exponential functions decrease to zero.
Causality in time space yield $t>0$ and because of that $e^{-\omega''t}$ decreases only for $\omega''>0$ to zero.
In summary causality in Fourier space means that the susceptibility is holomorphic in the upper complex plane ($\Im{\omega} = \omega'' > 0$).

Cauchy's integral theorem offers us the oppertunity to express the Fourier transformed susceptibility by a contour integral, where the arbitrary contour $\Gamma$ has to be taken in the upper complex plane or more presicly in the regime where $\chi_{\mt{AB}}(\omega)$ is holomorphic.
%
\begin{align}
	\chi_{\mt{AB}}(\omega) = \frac{1}{2\pi i} \oint\limits_{\Gamma} \dd{\zeta} \frac{\chi_{\mt{AB}}(\zeta)}{\zeta-\omega}
\end{align}
%
Our choice of the contour is some which goes from minus infity to infinity along the real part axis.
Along a semi circle in the upper half plane the contour is closed, see figure \todo{link to figure of contour}.
For reason of convergency the contour along the real part axis is moved in the upper half plane infinitesimal indicated with $i\eta$ where $\eta \to 0$ is implicated.

The contribution of the semi circle vanishs because $\chi_{\mt{AB}}(\omega)$ decreasing very fast for large values of $\omega$ is assumed.
Only a integral along the real part axis survives which can be evaluated by formally writing $\frac{1}{x+i\eta} = \mt{PV}\frac{1}{x}-i\pi\delta(x)$ where PV stands for taking the principal value.
%
\begin{align}
	\chi_{\mt{AB}}(\omega) &= \frac{1}{2\pi i} \int\limits_{-\infty}^{\infty} \dd{\omega'} \frac{\chi_{\mt{AB}}(\omega')}{\omega'-\omega-i\eta} 
	\notag \\
	\Leftrightarrow\ \chi_{\mt{AB}}(\omega) &= \frac{1}{2\pi i} \bigg[
		\mt{PV} \int\limits_{-\infty}^{\infty} \dd{\omega'} \frac{\chi_{\mt{AB}}(\omega')}{\omega'-\omega} 
		+ 
		i\pi \int\limits_{-\infty}^{\infty} \dd{\omega'} \chi_{\mt{AB}}(\omega') \delta(\omega'-\omega)
	\bigg]
	\notag \\
	\Leftrightarrow\ \chi_{\mt{AB}}(\omega) &= -\frac{i}{\pi} \mt{PV} \int\limits_{-\infty}^{\infty} \dd{\omega'} \frac{\Re{\chi_{\mt{AB}}(\omega')} + i\Im{\chi_{\mt{AB}}(\omega')}}{\omega'-\omega} 
	\notag \\
	\Leftrightarrow\ \chi_{\mt{AB}}(\omega) &= \frac{1}{\pi} \mt{PV} \int\limits_{-\infty}^{\infty} \dd{\omega'} \bigg[
		\frac{\Im{\chi_{\mt{AB}}(\omega')}}{\omega'-\omega}
		-i
		\frac{\Re{\chi_{\mt{AB}}(\omega')}}{\omega'-\omega} 
	\bigg]
\end{align}
%
In the second step one right hand side the complex susceptibility is written explicitly by her real and imaginary part.
Nothing keep us from doing this on the left side hand too and compare the real and imaginary parts of both sides respectively.
%
\begin{align}
	\Re{\chi_{\mt{AB}}(\omega)} &= \frac{1}{\pi} \mt{PV} \int\limits_{-\infty}^{\infty} \dd{\omega'} \frac{\Im{\chi_{\mt{AB}}(\omega')}}{\omega'-\omega}
	\\
	\Im{\chi_{\mt{AB}}(\omega)} &= -\frac{1}{\pi} \mt{PV} \int\limits_{-\infty}^{\infty} \dd{\omega'} \frac{\Re{\chi_{\mt{AB}}(\omega')}}{\omega'-\omega}
\end{align}
%
These two relations are called Kramer-Kronig-relation.
They take the real and imaginary part of the a function, here the susceptibility, in a very usefull relation.
In the later computation them are used to compute the Green function on the real axis out off the Green function on the imaginary axis and vice versa.
This is always needed if analytical continuation isn't possible, which is the case considering damping in the Green function.





















%
%
\subsection{Spectral representation}
%
%

























