\chapter{Calculation}
\label{ch: calculation}

In the last chapter the memory-matrix-formalism was introduced, which give us an exact formula to calculate correlation functions.
Now this formalism is used to determine the static conductivity of the spin-fermion-model, introduced in chapter (\todo{make link to chapter spin-fermion-model}), pertubate umklapp-scattering.


\section{Infinite conductivity in systems with unbroken translation symmetry}
\label{sec: Infinite conductivity in a system with unbroken translation symmetry}
%
%
After Drude published his theory about the electrical transport in metals \cite{Drude} in the beginning of the last century it is well known that a broken translation symmetry is needed to get a finite static conductivity.
Because of Neother's theorem it is also well known that a unbroken symmetry always implies a conserved quantity.
In the case of translation symmetry this quantity is the momentum.
Phenomenas breaking the translation symmetry are for example impurity scattering, electron-electron scattering and umklapp scattering.
Let us firstly assume the standard spin-fermion-model without a translation symmetry breaking pertubation.
In chapter (\todo{link to chapter spin-fermion-model}) it is showed that in the used model the momentum is conserved but the currunt isn't conserved.
This property is needed to calculate the static conductivity.

In general the static conductivity is given by taking the small frequencie limit of the conductivity and the conductivity itself is given by the current-current correlation function (J-J correlation function), which reslut directly from the linear response theory.
%
\begin{align}
	\sigma_{\mt{dc}} = \lim\limits_{z \to 0} \sigma(z) = \lim\limits_{z \to 0}\, \beta\,\mathcal{C}_{\mt{JJ}}(z)
\end{align}
%
The memory matrix formalism is used to calculate the J-J correlation function.
Before we attend us to the calculation of the correlation function we have to think about the set of operators introduced by defining the projection operator.
This set of operators has to be choosen for each calculation seperatly depending of the model and the quantity of interest.
In our case we choose only a set of two operators namly the momentum and the current, because we want to figure out the influnece of the momentum on the current.
That means the projector $\mathcal{P}$ projects into the two dimensional sub-Hilbertspace, spanned P and J.
Lets go back to the correlation function, defined by equation (\todo{reference to correlation function}), where the sum over $k$ und $l$ is implied.
Because our interest is focused on the J-J correlation function each index $j$ and $l$ is set to $J$.
%
\begin{align}
	\mathcal{C}_{\mt{JJ}}(z) = 
		\frac{i}{\beta} \bigg[ 
			z \delta_{i\mt{J}} 
			+ 
			i \beta 
			\obra{\dot{\mt{A}}_{i}} 
			\hat{Q} \frac{i}{z - \hat{Q} \hat{L} \hat{Q}} \hat{Q} 
			\oket{\dot{\mt{A}}_{k}}
			\chi_{k\mt{J}}^{-1}
		\bigg]^{-1}
		\chi_{i\mt{J}},
\end{align}
%
Now the sum over $k$ is performed explicitly where both contriubtions for J and P vanish.
Let us look seperatly on J and P starting with the latter case.
In our observed model the momentum is conserved and so the time deriviative of P is zero.
No more words are needed to see that the expactation value doesn't contribute.
In the case of J the time derivative doesn't vanish so the two dimensional sub-Hilbertspace and the action of J in this Hilbertspace has to be consider.
In the investigated system the whole current lives in the J-P Hilbertspace to any time, so no single part of J is transported out of the Hilbertspace.
The appering operator $\mathcal{Q}$ is the inverse of $\mathcal{P}$ and therefore projected out of the J-P Hilbertspace.
Combining both statements it is clear that $\mathcal{Q}\toket{\dot{\mt{J}}} = 0$.
The only resulting term choosing $i=J$ is
%
\begin{align}
	\mathcal{C}_{\mt{JJ}}(z) = \frac{i}{\beta} z^{-1} \chi_{\mt{JJ}}(\omega=0) = \frac{i}{z} \mathcal{C}_{\mt{JJ}}(t=0),
	\label{eq: correlation function unpertubated system}
\end{align}
%
where the correlation function at $t = 0$ is given by the scalar product $\tobraket{\mt{J}(0)}{\mt{J}(0)}$ defined in equation (\todo{reference to scalar product}).
During the motivation of the previous chapter we explained that each observalbe can be split in one secular and one non-secular part.
This is equatable with splitting a vector in a parallel and a perpendicular component, respectivily.
%
\begin{align}
	\oket{\mt{J}} = \oket{\mt{J}_{\mid\mid}} + \oket{\mt{J}_{\bot}}
	\label{eq: splitting current}
\end{align}
%
What does this mean in physical language?
In the investigated system the current isn't conserved, but nevertheless a part of him is it.
This part is represented by the secular part and has to be parallel with the momentum.
Therfore the projection from J at P yield the parallel component of J.
%Let us concentrate on the parallel component and bear in mind the perpendicular one for later.
%In the language of the memory matrix formalism the secular part is the projection of P on J.
%
\begin{align}
	\oket{\mt{J}_{\mid\mid}} = \mathcal{P}\oket{\mt{J}} = \frac{\odyad{\mt{P}}{\mt{P}}}{\obraket{\mt{P}}{\mt{P}}} \oket{\mt{J}} = \frac{\chi_{\mt{PJ}}}{\chi_{\mt{PP}}} \oket{\mt{P}}
	\label{eq: parallel current as projection}
\end{align}
%
Firstly this give us the oppertunity to write the J-J correlation function into two parts one parrallel and one perpendicular correlation function using equation \eqref{eq: splitting current}.
The mixed correlation functions are zero by construction because $\toket{\mt{J}_{\mid\mid}}$ and $\toket{\mt{J}_{\bot}}$ are orthogonal and therfore the scalar product of both is zero.
%
\begin{align}
	\mathcal{C}_{\mt{JJ}}(t=0) = \obraket{\mt{J}(0)}{\mt{J}(0)} = \obraket{\mt{J}_{\mid\mid}}{\mt{J}_{\mid\mid}} + \obraket{\mt{J}_{\bot}}{\mt{J}_{\bot}}
\end{align}
%
In a next step equation \eqref{eq: parallel current as projection} is used to write the parallel J-J correlation function in a expression depending on the P-P correlation function which is nothing else $\tobraket{\mt{P}}{\mt{P}}$.
%
\begin{align}
	\mathcal{C}_{\mt{JJ}}(t=0) = \frac{\vert\chi_{\mt{PJ}}\vert^{2}}{\vert\chi_{\mt{PP}}\vert^{2}} \mathcal{C}_{\mt{PP}}(t=0) + \obraket{\mt{J}_{\bot}}{\mt{J}_{\bot}}
\end{align}
%
Now let us insert back this expression into equation \eqref{eq: correlation function unpertubated system} which give us multipling by $\beta$ the conductivity
%
\begin{align}
	\sigma (z) = \frac{\vert\chi_{\mt{PJ}}\vert^{2}}{\vert\chi_{\mt{PP}}\vert} \frac{i}{z}  + \sigma_{\mt{reg}}(z)
\end{align}
%
where the regular conductivity $\sigma_{\mt{reg}}(z) = \frac{i \beta}{z} \obraket{\mt{J}_{\bot}}{\mt{J}_{\bot}}$ is introduced.
The physical meaning of $\sigma_{\mt{reg}}(z)$ is discussed in a view steps, if we have the final expression for the conductivity.
In the whole calculation there wasn't made a condition on $z$, so the equation for the conductivity is valid for each $z$ in the complex plane.
In reality the conductivity isn't depending on an complex frequency.
Physical quantities are always real.
Therefore we have to set $z = \omega + i \eta$, where $\omega \in \mathbb{R}$ and the limit $\eta \to 0$ is implied.
Using $\frac{1}{\omega + i\eta} = \mathcal{P}\frac{1}{\omega} - i\pi\delta(\omega)$ the conductivity is given by
%
\begin{align}
	\sigma(\omega) = \frac{\vert\chi_{\mt{PJ}}\vert^{2}}{\vert\chi_{\mt{PP}}\vert} \bigg( \mathcal{P} \frac{i}{\omega} + \pi \delta(\omega) \bigg) + \sigma_{\mt{reg}}(\omega)
	\label{eq: conductivity unpertubed system}
\end{align}
%
where in this special case $\mathcal{P}$ sympolizied that the prinzipal value is taken.
Equation \eqref{eq: conductivity unpertubed system} yield us exactly the expected result.
For small frequencies the main contribution is generated by the $\delta$-distribution, so the conductivity becomes infinity.
This isn't really surprising because the translation symmetry isn't broken in the investigated system.
If voltage is applied on a system like ours the electrons accelerate infinite long.
There is nothing they can scatter on and loss some momentum.
The electrons accelerate more and more and this results in an infinite conductivity.
Only in a system with broken translation symmetry it's possible for the electrons to loss some momentum by scattering with the lattice for example.
This results in a finite conductivity, so the $\delta$-peak becomes smaller.
The factor in front of the $\delta$-distribution is the so called Drude-wight.

Let us now talk about the regular part of the conductivity.
We don't want here to calculate some explicite expression, a small physical discussion about this part should be enough at this point.
In every physical system there are some kind of effects which are always there and it's nearly impossible to suppress them.
These effects are noise, fluctuations and other effects influenced by random forces.
All of them are summarized in the regular conductivity.






























