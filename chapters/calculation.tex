\chapter{Calculation}
\label{ch: calculation}

In the last chapter the memory-matrix-formalism was introduced, which give us an exact formula to calculate correlation functions.
Now this formalism is used to determine the static conductivity of the spin-fermion-model, introduced in chapter \todo{make link to chapter spin-fermion-model}, pertubate umklapp-scattering.


\section{Infinite conductivity in systems with unbroken translation symmetry}
\label{sec: Infinite conductivity in a system with unbroken translation symmetry}
%
%
After Drude published his theory about the electrical transport in metals \cite{Drude} in the beginning of the last century it is well known that a broken translation symmetry is needed to get a finite static conductivity.
Because of Neother's theorem it is also well known that a unbroken symmetry always implies a conserved quantity.
In the case of translation symmetry this quantity is the momentum.
Phenomenas breaking the translation symmetry are for example impurity scattering, electron-electron scattering and umklapp scattering.
Let us firstly assume the standard spin-fermion-model without a translation symmetry breaking pertubation.
In chapter \todo{link to chapter spin-fermion-model} it is showed that in the used model the momentum is conserved but the currunt isn't conserved.
This property is needed to calculate the static conductivity.

The static conductivity is given by the conductivity taking the limit for small frequencies and the conductivity itself is given by the current-current correlation function.
%
\begin{align}
	\sigma_{\mt{dc}} = \lim\limits_{z \to 0} \sigma(z) = \lim\limits_{z \to 0}\, \beta\,\mathcal{C}_{\mt{JJ}}(z)
\end{align}
%
In the notation of the memory-matrix-formalism the current-current correlation function ist given by
%
\begin{align}
	\mathcal{C}_{\mt{JJ}}(z) = 
		\frac{i}{\beta} \bigg[ 
			z \delta_{i\mt{J}} 
			+ 
			i \beta 
			\obra{\dot{\mt{A}}_{i}} 
			\hat{Q} \frac{i}{z - \hat{Q} \hat{L} \hat{Q}} \hat{Q} 
			\oket{\dot{\mt{A}}_{k}}
			\chi_{k\mt{J}}^{-1}
		\bigg]^{-1}
		\chi_{i\mt{J}},
\end{align}
%
where a sum over $k$ is implied.
%
\begin{align}
	\mathcal{C}_{\mt{JJ}}(z) = \frac{i}{\beta} z^{-1} \chi_{\mt{JJ}}(\omega=0) = \frac{i}{z} \mathcal{C}_{\mt{JJ}}(t=0)
\end{align}
%

%
\begin{align}
	\oket{\mt{J}} = \oket{\mt{J}_{\mid\mid}} + \oket{\mt{J}_{\bot}}
\end{align}
%

%
\begin{align}
	\oket{\mt{J}_{\mid\mid}} = \hat{P}\oket{\mt{J}} = \frac{\odyad{\mt{P}}{\mt{P}}}{\braket{\mt{P}}{\mt{P}}} \oket{\mt{J}} = \frac{\chi_{\mt{PJ}}}{\chi_{\mt{PP}}} \oket{\mt{P}}
\end{align}
%

%
\begin{align}
	\mathcal{C}_{\mt{JJ}}(t=0) = \obraket{\mt{J}(0)}{\mt{J}(0)} = \obraket{\mt{J}_{\mid\mid}}{\mt{J}_{\mid\mid}} + \obraket{\mt{J}_{\bot}}{\mt{J}_{\bot}}
\end{align}
%

%
\begin{align}
	\mathcal{C}_{\mt{JJ}}(t=0) = \frac{\vert\chi_{\mt{PJ}}\vert^{2}}{\vert\chi_{\mt{PP}}\vert^{2}} \mathcal{C}_{\mt{PP}}(t=0) + \obraket{\mt{J}_{\bot}}{\mt{J}_{\bot}}
\end{align}
%

%
\begin{align}
	\mathcal{C}_{\mt{JJ}}(z) = \frac{i}{z \beta} \frac{\vert\chi_{\mt{PJ}}\vert^{2}}{\vert\chi_{\mt{PP}}\vert} + \frac{i}{z} \obraket{\mt{J}_{\bot}}{\mt{J}_{\bot}}
\end{align}
%

%
\begin{align}
	\sigma (z) = \frac{\vert\chi_{\mt{PJ}}\vert^{2}}{\vert\chi_{\mt{PP}}\vert} \frac{i}{z}  + \sigma_{\mt{reg}}(z)
\end{align}
%
where the regular conductivity $\sigma_{\mt{reg}}(z) = \frac{i \beta}{z} \obraket{\mt{J}_{\bot}}{\mt{J}_{\bot}}$ is introduced.
Setting $z = \omega + i\eta$, where the limit $\eta \to 0$ is implied.
%
\begin{align}
	\sigma_{\mt{dc}} = \frac{\vert\chi_{\mt{PJ}}\vert^{2}}{\vert\chi_{\mt{PP}}\vert} \bigg( \mathcal{P} \frac{i}{\omega} + \pi \delta(\omega) \bigg)
\end{align}
%
where $\mathcal{P}$ sympolizied that the prinzipal value is taken.































